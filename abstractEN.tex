% !TeX spellcheck = en_GB
%%%%%%%%%%%%%%%%%%%%%%%%%% phdsymp_sample2e.tex %%%%%%%%%%%%%%%%%%%%%%%%%%%%%%
%% changes for phdsymp.cls marked with !PN
%% except all occ. of phdsymp.sty changed phdsymp.cls
%%%%%%%%%%              %%%%%%%%%%%%%
%%%%%%%%%% More information: see the header of phdsymp.cls %%%%%%%%%%%%%
%%%%%%%%%%              %%%%%%%%%%%%%
%%%%%%%%%%%%%%%%%%%%%%%%%%%%%%%%%%%%%%%%%%%%%%%%%%%%%%%%%%%%%%%%%%%%%%%%%%%%%%%


\documentclass[twocolumn]{phdsymp} %!PN

\usepackage[english]{babel}  % Voor nederlandstalige hyphenatie (woordsplitsing)

\usepackage{graphicx}     % Om figuren te kunnen verwerken
\usepackage{graphics}			% Om figuren te verwerken.

\graphicspath{images/}

\PassOptionsToPackage{hyphens}{url}
\usepackage{url}

\usepackage[T1]{fontenc}

\usepackage{amsmath}
\usepackage{packages/customcommands}

\hyphenation{}

\def\BibTeX{{\rm B\kern-.05em{\sc i\kern-.025em b}\kern-.08em
 T\kern-.1667em\lower.7ex\hbox{E}\kern-.125emX}}

\newtheorem{theorem}{Theorem}

\begin{document}

\title{Client-side evaluation of GeoSPARQL queries over heterogeneous data sources} %!PN

\author{Andreas De Witte}

\supervisor{dr. ing. Pieter Colpaert, dr. ir. Ruben Taelman, Brecht Van de Vyvere, Julian Andres Rojas Melendez}

\maketitle

\begin{abstract}
    On the web as it's currently known, users can easily understand pages of websites. This is not the case for computers, they need to do a lot of effort to substract both meaning and context from sentences. The web, as it is today, is not build to be interpreted by machines. Thanks to the Semantic Web, which is an extension to the current web, it is possible for machines to understand the pages of websites.

    It is currently possible to query for linked data to limited amount of data sources, because very few data sources support linked data. Theses queries are executed with SPARQL, which has different working implementations. For geographical queries, GeoSPARQL is needed. However, there are only few implementations of GeoSPARQL and most of these implementations are working incorrect.

    In this work, a limited implementation of GeoSPARQL is made in order to request data in RDF format and calculate the topological relations in this data. With this implementation, it has been tested for which interfaces these topological relations can be calculated on the client-side. Doing this on the client-side is important for many reasons. The major reason is that many of these calculations would cripple a server, while these calculations for only one user on the client-side is very possible. In other words, doing this on the client-side is an effective way of spreading the load. This paper gives new insights about handling geographical queries on the client-side.

    Like this, it appeared to be very simple to calculate the topological relations on the client-side when the data source is a TPF-interface. However, when the source is a SPARQL-endpoint, this is harder to achieve, but still possible. It's also possible to detect whether the source is a GeoSPARQL-endpoint. If this is the case, the entire query can be handled at the source.
    
    The conclusion can be made that handling these kind of queries is better on the client-side. like this, the entire query can be processed, even when the source doesn't fully support it. This scription is mostly useful for computer scientists who are true experts about Semantic Web. But it can also be used by enthousiasts for receiving a better understanding of the Semantic Web and it's possibilities.
\end{abstract}

\begin{keywords}
    Semantic Web, linked data, OGC, GeoSPARQL, client-side, topolocical relation
\end{keywords}

\section{Introduction}
hallo






\bibliographystyle{phdsymp}
%%%%%\bibliography{bib-file} % commented if *.bbl file included, as
%%%%%see below


%%%%%%%%%%%%%%%%% BIBLIOGRAPHY IN THE LaTeX file !!!!! %%%%%%%%%%%%%%%%%%%%%%%%
%% This is nothing else than the phdsymp_sample2e.bbl file that you would%%
%% obtain with BibTeX: you do not need to send around the *.bbl file  
%%
%%---------------------------------------------------------------------------%%
%
\begin{thebibliography}{1}
    \bibitem{berners2001semantic}
    Berners-Lee, Tim and Hendler, James and Lassila, Ora
    \newblock {\em The semantic web},
    \newblock Scientific american, vol. 284, no. 5, pp. 34–43, 2001.

    \bibitem{berners2006linkeddata}
    Berners Lee, Tim
    \newblock {\em Linked Data},
    \newblock 2006.

    \bibitem{lassila1998resource}
    Lassila, Ora and Swick, Ralph R and others
    \newblock {\em Resource description framework (RDF) model and syntax specification},
    \newblock 1998.

    \bibitem{heath2011linked}
    Heath, Tom and Bizer, Christian
    \newblock {\em Linked data: Evolving the web into a global data space},
    \newblock Synthesis lectures on the semantic web: theory and technology, vol. 1, no. 1, pp. 1–136, 2011.

    \bibitem{sparql2013querylanguage}
    Harris, Steve and Seaborne, Andy
    \newblock {\em SPARQL 1.1 Query Language},
    \newblock World Wide Web Consortium, 2013.

    \bibitem{taelman2018comunica}
    Taelman, Ruben and Van Herwegen, Joachim and Vander Sande, Miel and Verborgh, Ruben
    \newblock {\em Comunica: a modular SPARQL query engine for the web},
    \newblock in International Semantic Web Conference. Springer, 2018, pp. 239–255.

    \bibitem{ogcdocs}
    \newblock {\em Open Geospatial Consortium},
    \newblock URL: https://ogc.org

    \bibitem{geosparqlsupport}
    \newblock {\em GeoSPARQL support: What is GeoSPARQL},
    \newblock URL: http://graphdb.ontotext.com/documentation/standard/geosparql-support.html

    \bibitem{shen2018classification}
    Shen, Jingwei and Chen, Min and Liu, Xintao
    \newblock {\em Classification of topological relations between spatial objects in two-dimensional space within the dimensionally extended 9-intersection model},
    \newblock Transactions in GIS, vol. 22, no. 2, pp. 514–541, 2018.
\end{thebibliography}
%
%%---------------------------------------------------------------------------%%


\end{document}