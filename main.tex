% Master thesis template for Ghent University (2018)
%
%
%  !!!!!!!!!!!!!!!!!!!!!!!!!!!!!!!!!!!!!!!!!!!!!!!!!!!!!!!!!!!!
%  !!  MAKE SURE TO SET lualatex OR xelatex AS LATEX ENGINE  !!
%  !!!!!!!!!!!!!!!!!!!!!!!!!!!!!!!!!!!!!!!!!!!!!!!!!!!!!!!!!!!!
%  !! For overleaf:                                          !!
%  !!     1. click gear icon in top right                    !!
%  !!     2. select `lualatex` in "latex engine"             !!
%  !!     3. click "save project settings"                   !!
%  !!                                                        !!
%  !!!!!!!!!!!!!!!!!!!!!!!!!!!!!!!!!!!!!!!!!!!!!!!!!!!!!!!!!!!!
%
%
%  History
%    2014         Doctoral Thesis of Bruno Volckaert
%    2017         Adapted to master thesis by Jerico Moeyersons
%    2018         Cleanup by Merlijn Sebrechts
%
%  Latest version
%    https://github.com/galgalesh/masterproef-template
%
\documentclass[11pt,a4paper,twoside, openany]{book}
\usepackage[a4paper,includeheadfoot,margin=2.50cm]{geometry}

\setlength{\parindent}{0cm}           % indent of the first sentence of a paragraph
\setlength{\parskip}{1em}             % space between paragraphs
\renewcommand{\baselinestretch}{1.2}  % stretch horizontal space between everything

\usepackage{graphicx}
\graphicspath{{images/}}
\usepackage{pdfpages}
\usepackage{enumitem}
\usepackage{float}
\usepackage{caption}
\usepackage{subcaption}
\usepackage[toc,page]{appendix}
\usepackage[section]{placeins}
\makeatletter
\AtBeginDocument{%
	\expandafter\renewcommand\expandafter\subsection\expandafter{%
		\expandafter\@fb@secFB\subsection
	}%
}
\makeatother
\usepackage{svg}
\usepackage{amsmath}
\usepackage{graphicx}
\usepackage{epstopdf}
\epstopdfsetup{update} % only regenerate pdf files when eps file is newer
\usepackage[chapter]{minted}                           % for modern code highlighting
\newenvironment{code}{\captionsetup{type=listing}}{}   % To get multiline code fragments working: https://tex.stackexchange.com/a/53540/72273

\AtBeginEnvironment{minted}{\fontsize{9}{9}\selectfont}

\PassOptionsToPackage{hyphens}{url}
\usepackage{hyperref}
\usepackage{url}
\usepackage{quotchap}              % For the fancy quotes next to the chapter titles
\usepackage{wrapfig}               % Handy package to position images
\usepackage[numbers]{natbib}       % For bibliography; use numeric citations
\bibliographystyle{IEEEtran}
\usepackage[nottoc]{tocbibind}     % Put Bibliography in ToC

%
% Defines \checkmark to draw a checkmark
%
\usepackage{tikz}
\def\checkmark{\tikz\fill[scale=0.4](0,.35) -- (.25,0) -- (1,.7) -- (.25,.15) -- cycle;}

%
% For tables
%
\usepackage{booktabs}
\usepackage{array}
\usepackage{ragged2e}  % for '\RaggedRight' macro (allows hyphenation)
\newcolumntype{L}[1]{>{\raggedright\let\newline\\\arraybackslash\hspace{0pt}}m{#1}}
\newcolumntype{C}[1]{>{\centering\let\newline\\\arraybackslash\hspace{0pt}}m{#1}}
\newcolumntype{R}[1]{>{\raggedleft\let\newline\\\arraybackslash\hspace{0pt}}m{#1}}


%
% Own packages added
%
\usepackage[colorinlistoftodos]{todonotes}
\usepackage{xcolor}
\usepackage{tabularx}
\usepackage[acronym,toc]{glossaries} % moeilijke woorden en afkortingen + in toc plaatsen
\usepackage{titling}			   % For \thetitle and \theauthor
\usepackage{multirow} % multi row gebruiken in tables
\usepackage{titlesec, blindtext, color, colortbl}

\usepackage{packages/customcommands}
\usepackage{packages/customcolors}

\renewcommand*{\acronymname}{Lijst van afkortingen}






%
% Support for splitting Dutch words correctly
%
\usepackage{polyglossia}
\setdefaultlanguage[babelshorthands=true]{dutch}

%
% Add support for locally installed fonts and
% add a new command for the UGent font
%
\usepackage{fontspec}
\newfontfamily{\ugentfont}{UGent Panno Text}

% Manually specify additional hypnations for words
\hyphenation{heterogene}

%
% Translated strings. If these aren't set, the English words are used.
%
\addto\captionsenglish{%
  \renewcommand{\contentsname}%
    {Inhoudsopgave}%
}
\renewcommand\appendixtocname{Bijlagen}
\renewcommand\appendixpagename{Bijlagen}
\renewcommand{\listoflistingscaption}{Lijst van listings}
\newfloat{listing}{thp}{lop}
\floatname{listing}{Code}


\newcommand\foreign[1]{\emph{#1}}
\newcommand\inlinecode[1]{\emph{#1}}
%
% Set the title and your name
%
\title{Client-side evaluatie van GeoSPARQL opvragingen over heterogene gegevensbronnen}
\author{Andreas De Witte}
\include{special_chapters/woordenlijst} % lijst van speciale woorden
%
%  END OF HEADER
%  The actual latex document content starts here.
%
\begin{document}
\begin{titlepage}
% Use the main UGent font
{\ugentfont

  % Center the content vertically on the page
  \vspace*{0.37\textheight}

  % Title and author are shown huge
  \begin{huge}

    \thetitle

    \theauthor

  \end{huge}

  \vspace{2.8cm}

  % Promotoren en begeleiding
  \begin{Large}
    Promotoren: dr. ing. Pieter Colpaert, dr. ir. Ruben Taelman\\
    Begeleiding: Brecht Van de Vyvere, Julian Andres Rojas Melendez

    Masterproef ingediend tot het behalen van de academische graad van Master of Science in de Industriële Wetenschappen: informatica
  \end{Large}

  \vspace{1cm}

  % UGent footer
  \begin{Large}
    \begin{wrapfigure}{r}{0.15\textwidth}
      \includegraphics[width=0.15\textwidth]{ugent.png}
    \end{wrapfigure}

    Vakgroep Informatietechnologie\\
    Voorzitter: prof. dr. ir. Bart Dhoedt\\
    Faculteit Ingenieurswetenschappen en Architectuur\\
    Academiejaar 2019-2020
  \end{Large}
}
\end{titlepage}
           % Front matter
\newpage\thispagestyle{empty}\mbox{}  % White page
% !TeX spellcheck = nl_NL
\thispagestyle{empty}    % Don't show page number
\textbf{Dankwoord}

Na zes maanden werken is deze masterproef afgerond. Ik heb hier zeer veel aan gewerkt en zonder de hulp van verschillende personen zou het mij niet gelukt zijn.

Allereerst wil ik mijn promotor Pieter Colpaert bedanken voor de continue begeleiding en feedback. Daarnaast wil ik Pieter ook bedanken voor de motivatie en alle opportuniteiten die hij mij gegeven heeft. 

Graag zou ik ook mijn tweede promotor Ruben Taelman bedanken voor de vele technische uitleg en de veelvuldige feedback die mijn thesis tot een goed einde gebracht heeft.

Daarnaast wil ik ook graag Ruben Dedecker bedanken voor de tips over het schrijven, de gezamelijke debugsessies en de raad over het opzetten van de demonstraties.

Tevens zou ik graag mijn medestudenten Demian Dekoninck en Thomas Aelbrecht bedanken voor de steun bij het maken van deze masterproef. Ook van jullie heb ik zeer veel motivatie gekregen, alsook heeft jullie input verschillende van mijn problemen opgelost.

Ook zou ik graag mijn familie en vrienden bedanken om altijd klaar te staan voor mij in deze zware periode. Jullie zorgden zo nu en dan voor de nodige afleiding en ontspanning na een dag werken. Hierbij gaven jullie mij telkens de nodige aanmoediging om door te gaan.

Tot slot zou ik iedereen willen bedanken die mij geïnspireerd heeft tijdens het schrijven en iedereen die mijn thesis heeft nagelezen. 

Bedankt iedereen!

Andreas De Witte          % Word of thanks
\newpage\thispagestyle{empty}\mbox{}  % White page
\includepdf[pages={-}]{abstractNL.pdf}  % Extended Abstract not done yet
\includepdf[pages={-}]{abstractEN.pdf}  % Extended Abstract not done yet
\tableofcontents                      % Table of Contents


\listoffigures                        % List of figures
\listoftables                         % List of tables
\listoflistings                       % List of listings (code fragments)
\addcontentsline{toc}{chapter}{Lijst van listings} % List of listings in toc
\printglossary % moeilijke woorden tonen
\printglossary[type=\acronymtype] % afkortingen tonen
\listoftodos 						  % show list of all to do's

% Include the main chapters of the thesis below
%
\begin{savequote}[0.55\linewidth]
	``The future belongs to those who believe in the beauty of their dreams.''
	\qauthor{\textasciitilde 
    Eleanor Roosevelt}
\end{savequote}

\chapter{Inleiding}
Het Web zorgt ervoor dat heel wat informatie beschikbaar is voor mensen en gedeeld kan worden door mensen. De webpagina’s zijn dan ook makkelijk toegankelijk voor iedereen die beschikt over een internetverbinding. Helaas betekent dit niet dat machines de gegevens even makkelijk kunnen decoderen. Om dagelijkse taken uit te voeren is menselijke interactie dan ook essentieel. Wanneer we zonder tussenkomst van machines een daguitstap plannen dan is dit op zich al een zeer complex proces. Zo moet onder andere de agenda van alle betrokken personen vergeleken worden om te weten te komen wanneer (bijna) iedereen beschikbaar is. Er moet gecontroleerd worden of de weersverwachtingen ideaal zijn voor de uitstap. Ook moet er rekening gehouden worden met de interesses van de verschillende personen om te beslissen welk type uitstap gedaan zal worden. Indien we dit proces willen laten overnemen door \textit{intelligent agents} dan houdt dit per definitie in dat we machines toegang moeten kunnen verlenen tot allerhande persoonlijke informatie.

Verder zijn gegevens ook niet voor iedereen beschikbaar. Grote spelers zoals Google, Instagram, Linkedin hebben rechtstreeks toegang tot persoonlijke gegevens van mensen. Ze kunnen deze gegevens zelf bijhouden, opvragen en controleren. Wanneer verschillende bedrijven dezelfde gegevens van mensen gaan opslaan, ontstaan er duplicate data. Dit maakt het proces opnieuw complexer. Bovendien moet in deze ook rekening gehouden worden met de wetten op de privacy, die overigens per land verschillend zijn.  Een mogelijke oplossing om deze problemen te omzeilen is de verkregen info decentraliseren. Dit wil concreet zeggen dat de gebruiker zelf controle heeft over zijn gegevens. Bedrijven die info willen over bepaalde personen, zullen dit zelf bij de gebruikers moeten opvragen. De gegevens worden dus niet bijgehouden in echte databanken, maar in andere bronnen waarop verder in deze masterproef op ingegaan zal worden.

Bovendien is het moeilijk om te werken met geografische data, omdat hier weinig implementaties van gemaakt zijn. Het geografische is ook eerder wiskundig om op te lossen. Deze masterproef legt de focus op het werken met geografische gegevens.

\section{Overzicht}

In \chapterref{chap:literatuurstudie} wordt de \textit{state of the art} behandeld, waarbij in details ingegaan wordt op technologieën/technieken al bestaan. Hierbij geeft \sectionref{sec:semantic_web} uitleg over het Semantisch Web. Vervolgens leggen \sectionref{sec:linked_data}, \sectionref{sec:rdf} en \sectionref{sec:sparql} de gebruikte technologieën uit, namelijk Linked Data, RDF en SPARQL. \sectionref{sec:comunica} vertelt meer over Comunica. Comunica brengt de (net hiervoor) genoemde technologieën bij elkaar in een implementatie. \sectionref{sec:ogc} vertelt over het OGC. Dit is een organisatie die standaarden voorziet om met geografische data te werken. Ten slotte beschrijft \sectionref{sec:geosparql} een specifieke standaard van het OGC, namelijk GeoSPARQL. Hierbij is \sectionref{sec:geosparql} zo belangrijk omdat het onderzoek (zie \sectionref{sec:onderzoeksvraag}) een implementatie van GeoSPARQL vereist. Zo zal \sectionref{sec:geosparql} uitleggen waar rekening mee moet gehouden worden om de implementatie te maken.

In \chapterref{chap:implementatie} wordt de eigen implementatie uitgelegd van GeoSPARQL. Zo legt \sectionref{sec:impl_comunica} uit waarom Comunica zo een belangrijke rol speelt bij deze implementatie. Dit wordt gevolgd door \sectionref{sec:datastructuur}, \sectionref{sec:topologische_functies}, \sectionref{sec:niet_topologische_functies} en \sectionref{sec:projecties} die beschrijven welke keuzes gemaakt zijn voor de verschillende aspecten van de implementatie. Daarnaast beschrijft \sectionref{sec:testomgeving} hoe de testomgeving gemaakt is. Ten slotte wordt het voorgaande nogmaals overlopen in \sectionref{sec:impl_overzicht} om te verduidelijken hoe alles exact samenwerkt. Hierbij wordt ook aangehaald welke verbeteringen nog dienen gemaakt te worden in toekomstig werk.

Vervolgens wordt in \chapterref{chap:interfaces} beschreven hoe het effectieve testen van de onderzoeksvraag gebeurt. Hiervoor wordt de hierboven beschreven testomgeving gebruikt. In \sectionref{sec:testset} wordt uitlegd welke use-case en dataset gebruikt zijn voor het testen van dit onderzoek. Vervolgens wordt in \sectionref{sec:data-dump}, \sectionref{sec:impl_tpf_interface} en \sectionref{sec:impl_sparql_endpoint} gecontroleerd of de hypothesen (zie \sectionref{sec:onderzoeksvraag}) voldaan zijn.

Ten slotte zal in \chapterref{chap:conclusie} een uiteindelijke conclusie getrokken worden. Hier wordt geantwoord op de vraag welke ``Linked data publicatie''-interfaces uitgebreid kunnen worden met GeoSPARQL-functionaliteiten door de filtering op de client uit te voeren.


\section{Probleemstelling en doel}
\label{sec:probleemstelling_doel}
Het creëren van een Semantisch Web staat nog in zijn kinderschoenen, met al enkele jaren onderzoek op de teller. Hoewel er al veel vooruitgang geboekt is, vereist het nog steeds zeer veel werk. Het ophalen van gegevens op het internet is reeds mogelijk door \textit{query engines} zoals onder andere Comunica en Virtuoso. Er is echter een veel beperkter aanbod aan mogelijkheden om met geografische informatie te werken. De bestaande implementaties van GeoSPARQL zijn incompleet of niet voldoende meegaand met de regels die opgesteld zijn door het OGC. 

Bovendien is het met huidige implementaties niet mogelijk om te queryen over verschillende bronnen of bronnen van verschillende types. Een simpel voorbeeld om dit probleem te verduidelijken, is het volgende. Stel: de Belgische overheid heeft een dataset die de volledige grens van België beschrijft (aan de hand van OGC standaarden). Daarnaast bevat deze dataset een soortgelijke beschrijving van de gewesten, provincies, gemeenten, steden en wegen. Op die manier zou het mogelijk zijn om op te vragen welke gemeenten of steden binnen een een bepaalde provincie liggen. Daarnaast zou het mogelijk zijn om te vragen welke steden of wegen op een bepaalde afstand (of interessanter: een kleinere afstand) van een stad liggen. Veronderstel nu dat de Franse, Nederlandse en Duitse overheden beschikken over een gelijkaardige dataset. Dan zouden soortgelijke opvragingen in deze dataset gedaan kunnen worden. Het zou echter onmogelijk zijn te weten welke Franse (of Nederlandse of Duitse) steden op een bepaalde afstand van een Belgische stad liggen. 

Dit probleem zou niet voorkomen wanneer de techniek van Comunica uitgebreidt kan worden naar de functionaliteiten van GeoSPARQL. Deze masterproef zal een simpele dataset voorzien die een soortgelijk probleem als hierboven kan simuleren. Hierbij zal gebruik gemaakt worden van Comunica, om zo gebruik te maken van de reeds voorziene mogelijkheden om te queryen over heterogene interfaces. 


\section{Onderzoeksvraag}
\label{sec:onderzoeksvraag}
Zoals beschreven in \sectionref{sec:probleemstelling_doel} zal onderzocht worden in welke mate Comunica uitgebreid kan worden om de GeoSPARQL functionaliteiten te ondersteunen. Aangezien meerdere bronnen van verschillende types gebruikt kunnen worden, moet de filtering zelf op de client gebeuren. Dit wordt geformuleerd in een onderzoeksvraag die uiteindelijk in \chapterref{chap:conclusie} beantwoord zal worden. 

\textbf{Onderzoeksvraag} Welke ``Linked Data publicatie''-interfaces kunnen uitgebreid worden met GeoSPARQL-functionaliteiten door de filtering op de client uit te voeren?

Wanneer Comunica zelfstandig een bestand moet ophalen en vervolgens queryen, is het mogelijk dat dit bestand geografische gegevens bevat. Deze queries moeten afgehandeld kunnen worden op zo een manier dat topologische (en niet-topologische) relaties berekend kunnen worden. Deze bron bevat zelf geen logica en wordt vervolgens de \textit{baseline}.

\textbf{Hypothese 1} Het is mogelijk om GeoSPARQL queries uit te voeren over ``data dumps'' waarbij de filtering op de client-side gebeurt.

Wanneer de bron een \acrfull{tpf} interface is, is er een server die de gegevens aanbiedt. Hierbij is het opnieuw mogelijk dat de dataset geografische informatie bevat. Door het filteren op de server moet het wederom mogelijk zijn om GeoSPARQL opvragingen uit te voeren.

\textbf{Hypothese 2} Het is mogelijk om GeoSPARQL queries uit te voeren over ``\acrshort{tpf} interfaces'' door de filtering op de client uit te voeren.

Een dataset kan ook vrijgegeven worden aan de hand van een SPARQL \textit{endpoint}. Comunica heeft de functionaliteit om te queryen naar een SPARQL \textit{endpoint}. Hierbij is het niet vanzelfsprekend dat GeoSPARQL queries opgevraagd kunnen worden aan een SPARQL \textit{endpoint}. Dit is echter wel mogelijk door de filtering op de client-side te doen.

\textbf{Hypothese 3} Het uitvoeren van GeoSPARQL queries op een ``SPARQL \textit{endpoint}'' is niet vanzelfsprekend. Het is echter mogelijk door de filtering op de client uit te voeren.
\begin{savequote}[0.55\linewidth]
	``Knowledge is power.''
	\qauthor{\textasciitilde 
    Francis Bacon}
\end{savequote}

\chapter{Literatuurstudie}
\label{chap:literatuurstudie}
\section{Semantic Web}
In 2001 heeft Tim Berners-Lee (uitvinder van het World Wide Web) het over een nieuwe revolutie. Dit is de eerste introductie van het \textit{Semantic Web} (soms wordt er ook naar verwezen onder de term \textit{Web 3.0}). Hierbij zou het web zoals het toen was evolueren. Zo is het web altijd leesbaar geweest voor mensen, maar niet interpreteerbaar door machines. Het \textit{Semantic Web} zou hier verandering in brengen. Zo zouden machines het web op dezelfde manier kunnen interpreteren zoals mensen dat doen. Deze machines heten \textit{intelligent agents} en zij zouden complexe taken volledig autonoom kunnen uitvoeren \cite{berners2001semantic}.

Om dit mogelijk te maken zijn er verschillende stappen nodig. Als eerste moet er voor gezorgd worden dat er betekenis gegeven kan worden op een manier die computers kunnen begrijpen. Ook is het belangrijk dat deze kennis gerepresenteerd kan worden aan de machines. Zo wordt er gebruik gemaakt van het RDF model (zie \sectionref{sec:rdf}) met behulp van bijvoorbeeld XML. Verder is het ook belangrijk om er rekening mee te houden dat informatie uit verschillende databanken een andere terminologie kan gebruiken om hetzelfde uit te drukken. Hiervoor wordt gebruik gemaakt van verschillende ontologieën (een definitie van de term ontologie wordt gegeven in \subsubsectionref{subsubsec:ontology}). De echte kracht van het \textit{Semantic Web} zal komen wanneer programma's gemaakt worden die informatie verzamelen van verschillende bronnen (de zogenaamde \textit{intelligent agent}) \cite{berners2001semantic}.

Een belangrijk aspect om het \textit{Semantic Web} mogelijk te maken is dus decentralisatie. Hiermee wordt bedoeld dat de macht (hier in de vorm van informatie) niet in handen mag zijn van enkele grote spelers, maar verspreid moet worden. In een ideale vorm van het \textit{Semantic Web} zou elke persoon een \textit{pod} hebben die de informatie over zichzelf bevat. Wanneer een website toegang tot deze informatie zou willen, dan zou deze informatie uit de \textit{pod} opgehaald moeten worden. Dit zou nog andere voordelen brengen, zoals onder andere een verbeterde privacy (toegang verlenen aan wie de persoon wil).

\subsection{Semantic Web Stack}
Het is vanzelfsprekend dat de architectuur van het \textit{Semantic Web} gebaseerd is op een hiërarchie van talen, waarbij elke taal de mogelijkheden van de talen lager in deze hiërarchie optimaal zal benutten en uitbreiden. Deze hiërarchie is gevisualiseerd in \figureref{fig:semantic_web_stack}, ontworpen door Tim Berners-Lee. In de paper ``Semantic Web Architecture: Stack or Two Towers?'' worden alternatieve voorstellingen van de \textit{Semantic Web Stack} besproken \cite{horrocks2005semantic}. In deze scriptie wordt niet verder ingegaan op deze uitbreidingen. De lagen van de oorspronkelijke \textit{Semantic Web Stack} die belangrijk zijn voor deze scriptie, worden hieronder besproken.

\begin{figure}[ht]
    \centering
    \includegraphics[width=0.5\linewidth]{images/Semantic Web Stack.png}
    \caption{Semantic Web Stack (gebaseerd op \textit{Semantic Web Stack} \cite{semanticwebstack})}
    \label{fig:semantic_web_stack}
\end{figure}

\subsubsection{Unicode}
Unicode is een systeem voor het encoderen van karakters. Net zoals ASCII is het ontwikkeld met het doel om ontwikkelaars te ondersteunen in het maken van applicaties. Unicode pakt de problemen aan van eerdere karakter encodeer systemen, zoals onder het niet ondersteunen van alle karakters. Zo zal unicode een uniek nummer hebben voor elk karakter op elk platform, voor elk programma en in elke taal \cite{unicode}.

Unicode ligt aan de basis van de \textit{Semantic Web Stack} omdat het \textit{Semantic Web} documenten in verschillende talen moet kunnen doorgeven. Deze documenten moeten dus ook kunnen gerepresenteerd worden.

\subsubsection{URI}
URI staat voor \textit{Uniform Resource Identifier}. Dit is dus een uniforme manier voor het identificeren van objecten. Deze term wordt soms door elkaar gehaald met de term URL, wat staat voor \textit{Uniform Resource Locator}. Het grote verschil tussen beide is dat een URI een object kan identificeren (= hoe iets te benoemen), terwijl een URL een object kan localiseren (= waar iets te vinden). De verwarring tussen beide komt door de onderlinge relatie tussen beide. Om dit verschil te begrijpen is het belangrijk te weten dat de verzameling van URL's een subset is van alle URI's. Zo is elke URL een URI, maar niet omgekeerd \cite{uri}.

Naast unicode ligt URI mede aan de basis van de \textit{Semantic Web Stack} omdat deze het mogelijk maken om op het Web resources te identificeren op eenzelfde eenvoudige manier.

\subsubsection{XML}
XML staat voor \textit{Extensible Markup Language}. Het wordt gebruikt voor de beschrijving van data. Eén van de belangrijkste kenmerken van de XML standaard is dat deze op een zeer flexibele manier data kan structureren. Het W3C beveelt XML dan ook aan. XML werkt aan de hand van elementen die gedefinieerd worden door \textit{tags}. Zo heeft elk element een begin -en een eind\textit{tag}. XML ondersteunt ook geneste elementen zodat echte hiërarchiën gemaakt kunnen worden. XML is belangrijk vanwege zijn eenvoud en uitbreidbaarheid \cite{bray2000extensible}. 

\subsubsection{Namespaces}
XML Namespaces worden ook aanbevolen door het W3C. De reden hiervoor is om te voorkomen dat verschillende elementen dezelfde en dus conflicterende namen hebben. Op deze manier wordt de woordenschat gedifferencieerd, zodat deze woordenschat herbruikt kan worden. Het idee van namespaces steunt volledig op de werking van URI \cite{bray1999namespaces}.

\subsubsection{RDF Model, Syntax en Schema}
RDF staat voor \textit{Resource Description Framework}. RDF zal op een beschrijvende manier informatie geven. RDF is echter te belangrijk om kort besproken te worden en zal dus uitvoerig besproken worden in \sectionref{sec:rdf}.

\subsubsection{Ontology}
\label{subsubsec:ontology}
Het woord \textit{ontology} zorgt voor veel verwarring en heeft bijgevolg al meerdere verschillende definities gekregen. In zijn artikel ``What is an ontology?'' beschrijft Tom Gruber een ontologie als een specificatie van een conceptualisatie. De term ontologie komt van de filosofie waar het de betekenis heeft van een systematisch teken van het bestaan. Een ontologie kan beschreven worden als het definiëren van een set van representerende termen. Zo zullen relaties tussen objecten beschreven worden in een vorm die begrijpbaar is door mensen. Formeel betekent die dat een ontologie een uitspraak is van een logische theorie \cite{gruber2018ontology}. 

In de computerwetenschappen refereert de term ontologie naar een formele beschrijving van kennis. Zo kan informatie die van verschillende bronnen komt vertrouwen op de ontologieën om een gelijkaardige betekenis te krijgen.
\newpage
\section{Linked Data}

Het semantisch web gaat echter niet enkel over het plaatsen van data op het web. Het belangrijkste aspect van het semantisch web is het maken van links, zodat zowel personen als machines het web van data kunnen doorkruisen. Het belangrijke van gelinkte data is hetvolgende: wanneer je data hebt, kan je er andere gerelateerde data mee vinden. Bij gelinkte data worden deze links beschreven aan de hand van RDF. Hierbij worden URI's gebruiks voor het identificeren van objecten. Om data te interconnecteren zijn er vier regels, met als doen dat de informatie in de toekomst op onvoorspelbare manieren herbruikt zou kunnen worden. Daarnaast is het belangrijk dat de data open en toegankelijk is om herbruikt te worden \cite{berners2006linkeddata}. 

\subsection{Regels}
De eerste regel is om dingen te identificeren met URI's. Dit is nodig om te kunnen spreken over een semantisch web \cite{berners2006linkeddata}. 

De tweede regel is het gebruiken van HTTP URI's. Deze tweede regel is nodig zodat andere gebruikers de namen zouden kunnen opzoeken \cite{berners2006linkeddata}. 

De derde regel is dat bijhorende informatie gevonden moet kunnen worden wanneer een URI gevolgd wordt. Dit is in het basis formaat van RDF en XML. Deze kan ook doorzocht worden aan de hand van SPARQL (verder besproken in \sectionref{sec:sparql}), dit is een query service voor gelinkte data in RDF formaat \cite{berners2006linkeddata}. 

De vierde regel is het voorzien van links naar andere locaties die gelijkaardige data bevat, zodat dit opgezocht kan worden. Deze laatste regel is belangrijk om de informatie op het web te connecteren \cite{berners2006linkeddata}.

\subsection{Vijf sterren model}
Het vijf sterren model is een manier om informatie in te delen in hoeverre ze open is. Meer sterren betekent dat de informatie meer open is. Tim Berners-Lee stelde dit model voor als schema voor gelinkte open data. Gelinkte open data is namelijk een essentiëel onderdeel van het semantisch web.

Eén ster stelt hetvolgende: ``\textit{Available on the web but with an open licence, to be Open Data}''. Dit betekent dat gebruikers informatie kunnen ophalen, gebruiken en delen met iedereen. Het gaat hier echter louter over het delen van informatie, het maakt dus niet uit in welk formaat dit komt \cite{berners2006linkeddata}. 

Twee sterren stelt dan weer: ``\textit{Available as machine-readable structured data}''. Om twee sterren te krijgen is het belangrijk dat de informatie een bepaalde structuur heeft, zodat machines deze informatie kunnen verwerken. Dit kan bijvoorbeeld zijn in de vorm van een excel spreadsheet. Dit soort informatie is echter nog steeds vrij gesloten aangezien de gebruikers afhankelijk zijn van bepaalde software om toegang te krijgen tot de informatie \cite{berners2006linkeddata}.

Drie sterren betekent: ``\textit{The same as 2 stars, plus non-proprietary format}''. Het verschil om van twee sterren naar drie sterren te stijgen is het vermijden van de nood aan specifieke software om de informatie te bemachtigen. Dit kan bijvoorbeeld door de informatie op te slaan in CSV formaat \cite{berners2006linkeddata}.

Vier sterren is vervolgens: ``\textit{All the above plus, use open standards from W3C to identify things, so that people can point at your stuff}''. Voor het verdienen van de vierde ster moet de informatie voldoen aan de open standaarden van W3C. Zo moet het dingen identificeren aan de hand van RDF of SPARQL. Hierbij is het belangrijk dat gebruikers (aan de hand van URI) kunnen verwijzen naar de data \cite{berners2006linkeddata}.

Tenslotte betekent vijf sterren hetvolgende: ``\textit{All the above, plus: Link your data to other people’s data to provide context}''. Om de laatste ster ook te kunnen behalen, dient men de informatie te linken naar bijhorende informatie in een andere context. Op deze manier worden de links verder verspreidt. Hier wordt er dus letterlijk verwezen naar andere locaties, met als doel om meer context terug te vinden \cite{berners2006linkeddata}.

\subsection{Linked data gevisualiseerd}

\begin{figure}[ht!]
    \centering
    \includegraphics[width=\linewidth]{Linked-Data-Example.png}
    \caption{Linked Data voorbeeld}
    \label{fig:linked_data_example}
\end{figure}

Een voorbeeld van hoe het \textit{World Wide Web} eruit zou kunnen zien is geschetst in \figureref{fig:linked_data_example}. Dit voorbeeld toont de ideale situatie waar personen een eigen pod met informatie hebben. Zo hebben Alice en Bob elk hun eigen plaats op het web, waar informatie over hun te vinden is. Deze informatie zou onder andere hun naam, telefoonnummer, adres, interesses, werkomgeving, etc kunnen zijn. Daarnaast zijn er ook connecties tussen Alice en Bob. Om te beginnen is Bob gekend door Alice, waardoor er een verwijzing is naar meer context over Bob in zijn pod. Daarnaast wonen ze beide in dezelfde stad, waardoor het mogelijk is om bijvoorbeeld te zoeken naar iedereen die in een bepaalde stad woont. Al deze personen zullen terug te vinden zijn aan de hand van een verwijzing naar meer context (lees: meer informatie) over deze personen. Dit is echter een vereenvoudigd voorbeeld, in de reële situatie zijn er veel meer links en dit in alle richtingen.
\newpage
\section{RDF}
\label{sec:rdf}

RDF staat voor \textit{Resource Description Framework}. Het \textit{World Wide Web} is gemaakt voor mensen, en hoewel machines het kunnen lezen kunnen ze het niet altijd interpreteren. Het doel van RDF is om een algemene methode te voorzien om relaties tussen data objecten te beschrijven. Zo is RDF ontstaan in een poging om metadata te maken. Metadata worden gezien als data over data, maar kunnen beter geïnterpreteerd worden als data die \textit{web resources} beschrijven. RDF blijkt een zeer effectieve manier om informatie van verschillende bronnen te kunnen integreren door de informatie los te koppelen van zijn schema. Op deze manier kunnen de gegevens ook tegelijkertijd opgezocht worden. Zo poogt het dus om informatie op het web interpreteerbaar te maken voor machines. RDF steunt op de bestaande web standaarden zoals XML en URI. XML is echter slechts een mogelijke syntax. Er zijn verschillende andere manieren mogelijk om dezelfde RDF data te representeren. Het algemeen doel van RDF is het definiëren van een mechanisme. Dit mechanisme zorgt voor het beschrijven van \textit{resources} die geen veronderstellingen maken van een specifiek domein, noch een semantiek definiëren \cite{lassila1998resource}.

\subsection{RDF data model}
\begin{figure}
    \centering
    \includegraphics[width=0.5\linewidth]{images/spo.png}
    \caption{RDF statement}
    \label{fig:spo}
\end{figure}

De onderliggende structuur van een RDF uitdrukking is een collectie van triples. Elk van deze triples bestaat uit een \textit{subject} (= onderwerp), \textit{predicate} (= eigenschap) en \textit{object} (= voorwerp). Zoals te zien is in \figureref{fig:spo}, kan dit geïllustreerd worden als een \textit{node-arc-node link} (\textit{node} is een knoop, terwijl \textit{arc} een tak is). Deze collectie van triples kan bijgevolg gezien worden als een graaf. Hierbij is de richting van de \textit{arc} belangrijk, deze wijst in de richting van het \textit{object}. \cite{klyne2009resource}.

Eén enkele triple weerspiegelt een eenvoudige zin. Bij een kort terugblikken naar \figureref{fig:linked_data_example} is hetvolgende triple te zien: (\textit{Alice - Knows - Bob}). Deze triple staat letterlijk voor de zin ``\textit{Alice knows Bob}''. Deze zin hoeft echter niet altijd een exacte vertaling te zijn, bij een ander voorbeeld zien we dat er enkele korte woorden toegevoegd moeten worden om een gramaticaal correcte zin te bekomen: (\textit{Alice - Name - ``Alice''}) wordt dan weer ``Alice has name Alice''. In dit voorbeeld gaat het nu over de naam, maar het kan hier evenwel over een emailadres of een leeftijd gaan.

\subsubsection{URI-gebaseerde vocabulair}
Een \textit{node} kan een URI, een \textit{literal} of \textit{blank} zijn. Een URI referentie of een \textit{literal} die gebruikt wordt als \textit{node} identificeert waar de \textit{node} voor staat. Een URI referentie die gebruikt wordt als \textit{predicate} beschrijft dan weer de relatie tussen de ``dingen'' in de \textit{nodes} die geconnecteerd worden. Een \textit{blank node} is een \textit{node}, die louter staat voor een unieke code die gebruikt kan worden in één of meer RDF uitdrukkingen \cite{klyne2009resource}.

\subsubsection{Literals}
\textit{Literals} worden gebruikt om waarden zoals nummers en datums te identificeren. Elke \textit{literal} kan echter ook voorgesteld worden door een URI, maar vaak is het intuïtiever om een \textit{literal} te gebruiken. Een \textit{literal} kan enkel in het \textit{object} van de RDF uitdrukking staan, dus niet in het \textit{subject} of \textit{predicate}. Er bestaan twee soorten \textit{literals} \cite{klyne2009resource}:
\begin{enumerate}
    \item \textbf{\textit{Plain literal}}: deze \textit{literal} staat voor een \textit{string} die gecombineerd is met een optionele taal \textit{tag}. Deze wordt gebruikt om gewone tekst weer te geven en eventueel bij te plaatsen in welke taal deze tekst is.
    \item \textbf{\textit{Typed literal}}: deze \textit{literal} staat voor een \textit{string} die gecombineerd is met een \textit{datatype} URI. Deze URI wordt gebruikt om aan te duiden hoe deze informate geïnterpreteerd moet worden.
\end{enumerate}

Twee literals zijn gelijk indien alle volgende regels voldoen \cite{klyne2009resource}:
\begin{itemize}
    \item Beide strings zijn identiek;
    \item Ofwel hebben beide of geen van beide taal tags;
    \item Als ze taal tags hebben moeten deze identiek zijn;
    \item Ofwel hebben beide of geen van beide \textit{datatype} URIs;
    \item Als ze \textit{datatype} URIs hebben moeten deze identiek zijn.
\end{itemize}

\subsection{RDF serialisatie formaat}
\label{subsec:rdf_format}
Het is belangrijk te onthouden dat RDF geen data formaat is, maar een data model. Het is een beschrijving dat de gegevens zich moeten voorstellen in de vorm van (\textit{subject, predicate, object}) triples. Alvorens met een RDF graag kan publiceren, zullen de data geserialiseerd moeten worden, gebruik makend van een RDF syntax. De W3C verschillende formated gestandardiseerd, deze zijn hieronder vermeld. Er zijn zijn welliswaar nog meer mogelijkheden. Bij elk van deze mogelijkheden zal hetzelfde voorbeeld telkens herschreven worden in een ander formaat \cite{heath2011linked}.

Om de verschillende formaten duidelijk te maken (gebruik makend van principes zoals URIs, \textit{literals} zowel met als zonder \textit{datatype}) zal er bij de verschillende formaten éénzelfde voorbeeld uitgeschreven staan. Dit voorbeeld gaat over hoe het profiel van een persoon eruit zou kunnen zien. Aangezien het Turtle formaat het meest leesbare is, is enkel hier de uitgebreide versie zichtbaar. Bij andere formaten is dit profiel sterk ingekort. Bij dit voorbeeld zijn ook verschillende ontologieën gebruikt, zoals onder andere ``foaf'' en ``dbo''. Bovendien is bij dit voorbeeld ook te zien dat de \textit{predicate} ``a'' gebruikt wordt. Dit is een alternatief voor ``rdf:type'', maar verder volledig equivalent.

\subsubsection{RDF/XML}
De RDF/XML syntax is gestandaardiseerd door het W3C en is wijd gebruikt om Linked Data te publiceren op het web. Deze syntax wordt echter gezien als moeilijk te lezen en te schrijven voor mensen, waardoor deze steeds minder vaak gebruikt wordt. Bij deze syntax wordt het RDF-datamodel voorgesteld aan de hand van XML \cite{manola2004rdf}.

Een ingekorte versie van het voorbeeld beschreven in \subsectionref{subsec:rdf_format} in RDF/XML formaat is te zien in \listingref{listing:profile_xml}.

\begin{listing}[ht]
    \inputminted{xml}{data/profile_short.rdf}
    \caption{Profile in RDF/XML}
    \label{listing:profile_xml}
\end{listing}


\subsubsection{RDFa}
RDFa is dan weer een serialisatie formaat dat de RDF triples zal integreren in HTML documenten. In eerdere pogingen om RDF en HTML te mixen werden de RDF triples geïntegreerd in de \textit{comments}. Dit is hierbij niet het geval. Bij RDFa zijn de RDF triples verweven in de HTML DOM (= \textit{Document Object Model}). Dit betekent dat de bestaande inhoud van de pagina's aangeduid wordt met RDFa door de HTML code aan te passen. Hierdoor worden de gestructureerde data blootgesteld aan het web \cite{adida2012rdfa}.

Een ingekorte versie van het voorbeeld beschreven in \subsectionref{subsec:rdf_format} in RDFa formaat is te zien in \listingref{listing:profile_rdfa}.

\begin{listing}[ht]
    \inputminted{html}{data/profile_short.html}
    \caption{Profile in RDFa}
    \label{listing:profile_rdfa}
\end{listing}


\subsubsection{Turtle}
Turtle is een \textit{plain text} formaat voor de serialisatie van RDF-gegevens. Turtle voorziet prefixen voor \textit{namespaces} en andere verkortingen. Zo worden de prefixen bovenaan geschreven en moet elk triple eindigen op een ``.'', ``;'' of ``,''. Een ``.'' betekent dat de volgende triple volledig los staat van de huidige triple. Een ``;'' betekent dat de volgende triple hetzelfde \textit{subject} heeft als de huidige triple, waardoor er slechts twee waarden (\textit{predicate} en \textit{object}) op de volgende lijn staan. tenslotte betekent een ``,'' dat de volgende triple hetzelfde \textit{subject} en \textit{predicate} heeft als de huidige triple, waardoor er slechts één waarde (\textit{object}) op de volgende lijn staat. Deze verkortingen zijn echter geen verplichting. Aangezien Turtle zowel zeer leesbaar als schrijfbaar is, wordt deze in de meeste visuele teksten gebruikt. Vanwege de leesbaarheid zal dit formaat in de rest van deze scriptie (op de bijlagen na) ook gebruikt worden  \cite{beckett2014rdfturtle}.

Een uitgebreide versie van het voorbeeld beschreven in \subsectionref{subsec:rdf_format} in Turtle formaat is te zien in \listingref{listing:profile_turtle}.

\begin{listing}[ht]
    \inputminted{turtle}{data/profile.ttl}
    \caption{Extended profile in Turtle}
    \label{listing:profile_turtle}
\end{listing}

\subsubsection{N-Triples}
Het N-Triples-formaat is een subset van Turtle. Hierbij zijn de \textit{features} zoals prefixen en verkortingen weggelaten. Het valt het op dat dit serialisatie formaat veel redundantie heeft, zoals alle URIs die in elk triple volledig moeten worden gespecifieerd. Hierdoor zijn deze N-Triples-bestanden veel groter dan overeenkomende Turtle-bestanden. Naast het nadeel van grotere bestanden heeft deze redundantie ook een zeer groot voordeel. Dankzij de redundantie is het mogelijk om N-Triples-bestanden lijn per lijn te overlopen, waardoor het ideaal is om bestanden die te groot zijn om volledig in het geheugen te laden te verwerken. Daarnaast zijn N-Triples ook zeer ontvankelijk voor compressie, waardoor het netwerk verkeer gereduceerd wordt bij het uitwisselen van bestanden. Het N-Triples-formaat is zo de standaard om zeer grote dumps van Linked Data uit te wisselen (bijvoorbeeld voor backup doelen) \cite{beckett2014rdfntriples}.

Een ingekorte versie van het voorbeeld beschreven in \subsectionref{subsec:rdf_format} in N-Triples-formaat is te zien in \listingref{listing:profile_ntriples}. Hierbij zijn de lijnen gesplitst zodat deze op het blad zouden passen.

\begin{listing}[ht]
    \inputminted{turtle}{data/profile_short.nt}
    \caption{Profile in N-Triples}
    \label{listing:profile_ntriples}
\end{listing}


\subsubsection{JSON-LD}
JSON-LD staat voor JSON-LinkedData en is een \textit{lightweight} Linked Data formaat. JSON-LD is makkelijk leesbaar en schrijfbaar. Het is gebaseerd op het al langer bestaande JSON formaat. Aangezien JSON al langer gebruikt wordt om data door te geven, is JSON-LD het ideale formaat om Linked Data door te geven in een programmeeromgeving. Aangezien het dezelfde syntax heeft als JSON, kan het zonder andere software te installeren onmiddellijk gebruikt worden om RDF data te parsen en te manipuleren. Omdat JSON-LD zo handig in gebruik is, zal dit het meest gebruikte formaat zijn bij de implementaties die gemaakt zijn bij deze masterproef \cite{sporny2012json}.

Een (nog sterker) ingekorte versie van het voorbeeld beschreven in \subsectionref{subsec:rdf_format} in JSON-LD formaat is te zien in \listingref{listing:profile_jsonld}.

\begin{listing}[ht]
    \inputminted{json}{data/profile_short.jsonld}
    \caption{Profile in JSON-LD}
    \label{listing:profile_jsonld}
\end{listing}
\newpage
\section{SPARQL}
\label{sec:sparql}

SPARQL komt oorspronkelijk van ``\textbf{S}imple \textbf{P}rotocol \textbf{A}nd \textbf{R}DF \textbf{Q}uery \textbf{L}anguage'', maar aangezien het te uitgebreid werd om nog ``\textit{simple}'' te noemen, is het veranderd naar het recursieve acroniem ``\textbf{S}PARQL \textbf{P}rotocol \textbf{A}nd \textbf{R}DF \textbf{Q}uery \textbf{L}anguage'' \cite{sparql2011acronym}. Dit betekent letterlijk dat SPARQL een zoektaal is voor de opzoeking van RDF gebaseerde gegevens. Hierdoor is het ook zeer makkelijk om meerdere bronnen met RDF gegevens te combineren. Voor het gebruiksgemak te verhogen is SPARQL zeer gelijkaardig aan het meer bekende SQL. Om de vergelijking met SQL even door te trekken, kan RDF-data beschouwd worden als een soort tabel met drie kolommen: de \textit{subject} kolom, de \textit{predicate} kolom en de \textit{object} kolom. Hierbij zou het \textit{subject} analoog zijn aan een entiteit bij SQL. Het \textit{predicate} zou dan opnieuw staan voor welk veld (dus de kolom in de SQL tabel) een waarde heeft en het \textit{object} zou de effectieve waarde van dat veld zijn \cite{sparql2013querylanguage}. 

Er zijn echter ook belangrijke verschillen met SQL. Zo is de waarde van het \textit{object} vaak geïmpliceerd door de \textit{predicate} waarde. Daarnaast kunnen er ook meerdere (verschillende) \textit{object} waarden zijn voor hetzelfde \textit{predicate}, om zo een lijst te bekomen \cite{sparql2013querylanguage}.

Bovendien is SPARQL de \acrshort{w3c} aanbeveling als RDF query taal. De traditionele manier om een SPARQL \textit{query processor} te implementeren is door deze te gebruiken als interface voor een onderliggende databank. Dit wordt een SPARQL \textit{endpoint} genoemd. Dit is ook weer te vergelijken met hoe een SQL interface toegang geeft tot een relationele databank.

\subsection{SPARQL basisvoorbeelden}

Aangezien SPARQL gebruikt wordt om RDF gebaseerde gegevens op te vragen, zullen de \textit{statements} in de query zelf ook een RDF vorm hebben. Deze queries zijn syntactisch zeer gelijkend op het Turtle formaat. Zo is de simpelste vorm van een SPARQL query te zien in \listingref{listing:basic_sparql_query}. Deze zal letterlijk alle triples opvragen en deze weergeven. 

\begin{listing}[ht]
    \begin{minted}{sparql}
        SELECT *
        WHERE { ?s ?p ?o. }
    \end{minted}
    \caption{Basic SPARQL query}
    \label{listing:basic_sparql_query}
\end{listing}

Er zijn drie variabelen (namelijk ``?s'', ``?p'' en ``?o''), die achtereenvolgens weergegeven worden. Elk van deze variabelen kan ook aangepast worden naar een literal of een URI, om zo de query meer zinvol te maken. Zo is \listingref{listing:find_people_sparql_query} een meer zinvol voorbeeld. Hierin zal gekeken worden naar alle personen in de dataset (via foaf:Person). Vervolgens zal de naam en het emailadres opgehaald worden en in een variabele geplaatst worden, om zo ten slotte deze naam en email te laten zien. Alles bij elkaar zal \listingref{listing:find_people_sparql_query} van alle personen in de dataset de naam en het emailadres laten zien.

\begin{listing}[ht]
    \begin{minted}[xleftmargin=\parindent,linenos]{sparql}
        PREFIX foaf: <http://xmlns.com/foaf/0.1/>
        SELECT ?name ?email
        WHERE {
          ?person a foaf:Person.
          ?person foaf:name ?name.
          ?person foaf:mbox ?email.
        }
    \end{minted}
    \caption{Find people}
    \label{listing:find_people_sparql_query}
\end{listing}

Het voorbeeld uit \listingref{listing:find_people_sparql_query} kan uiteraard nog ingekort worden, zo kan ``?person'' nog verwijderd worden op lijn 5 en lijn 6, door op lijn 4 en lijn 5 de ``.'' te veranderen door een ``;''. Daarnaast heeft SPARQL ook \textit{blank nodes} (voorgesteld door ``[]''). Deze doen zich voor als variabelen, maar zijn het eigenlijk niet. Op deze manier kan zelfs de ``?person'' van lijn 4 weggelaten worden, door deze lijn tussen vierkante haken te plaatsen. Dit is mogelijk omdat de inhoud van ``?person'' niet getoond wordt, dus is deze niet echt nodig. Bij deze laatste vorm kan er bij de ``SELECT'' statement ook een ``*'' geplaatst worden, omdat er maar twee variabelen meer zijn. Zo krijgen we toch nog steeds de verwachte uitkomst van de query. Dit komt echter de leesbaarheid van het geheel niet ten goede, waardoor er meestal toch gekozen wordt voor de lange versie \cite{sparql2013querylanguage}.

\subsection{SPARQL functies}
\subsubsection{Functies}
Verder ondersteunt SPARQL verschillende functies, waaronder filterfuncties, om zo onder andere verder onderscheid te maken tussen welke lijnen al dan niet tot het resultaat mogen behoren. Deze functies kunnen bijvoorbeeld de waarde van een variabele veranderen of twee variabelen samenvoegen tot een nieuwe variabele. De filterfuncties kunnen dan weer bijvoorbeeld kijken naar de waarde of deze overeenkomt met een regex. Om uitbreidingen op SPARQL (zoals GeoSPARQL, besproken in \sectionref{sec:geosparql}) aan te brengen, zal het meeste werk zijn om nieuwe functies aan te brengen \cite{sparql2013querylanguage}.

\subsubsection{Matching alternatieven}
Naast filterfuncties zijn er ook verschillende alternatieven, zoals het gebruik van onder andere ``UNION'' (zeker niet te verwarren met de union functie van GeoSPARQL, besproken in \sectionref{sec:geosparql}) \cite{sparql2013querylanguage}.

\subsubsection{Negaties}
In SPARQL is het zelfs mogelijk om negaties te gebruiken (wat nog steeds lijkt op SQL), zoals een ``FILTER NOT EXISTS'' of ``MINUS''. Er is echter wel een zeer groot verschil tussen deze twee opties. Hierbij zal ``FILTER NOT EXISTS'' kijken of de waarden gelijk zijn, zodat deze verwijderd kunnen worden. De ``MINUS'' optie zal dan weer kijken of er een \textit{binding} (= gelijke variabele) aanwezig is tussen de twee gescheiden delen. Indien niet zal ``MINUS'' niets verwijderen \cite{sparql2013querylanguage}.

\subsection{SPARQL aggregaties}
Net zoals SQL heeft SPARQL ook de mogelijkheid om meerdere rijen te aggregeren. Hiervoor wordt gebruik gemaakt van de syntax ``GROUP BY''. De mogelijke aggregaatsfuncties zijn dan onder andere ``COUNT'', ``SUM'', ``MIN'', ``MAX'' en ``AVG'' \cite{sparql2013querylanguage}.

\subsection{SPARQL modifiers}
SPARQL voorziet nog vele functionaliteiten. Enkele voorbeelden zijn \cite{sparql2013querylanguage}:
\begin{itemize}
    \item ``ORDER BY'' voor het orderen van de uitkomst.
    \item ``DISTINCT'' om unieke resultaten te bekomen.
    \item ``LIMIT'' om aan te geven de hoeveel eerste rijen teruggegeven mogen worden.
\end{itemize}

\subsection{SPARQL query forms}
SPARQL heeft vier verschillende query vormen. Deze vormen zullen beslissen hoe het resultaat van een query eruit ziet. deze vormen zijn \cite{sparql2013querylanguage}:
\begin{itemize}
    \item ``SELECT'' wordt gebruikt om alle of een deel van de variabelen uit de query weer te geven.
    \item ``CONSTRUCT'' dient dan weer om een geldige RDF-graaf te construeren.
    \item ``ASK'' is de meest eenvoudige vorm en wordt gebruikt om te controleren of er een resultaat is voor een bepaalde query. Deze geeft dus een \textit{boolean} waarde terug.
    \item ``DESCRIBE'' geeft een RDF-graaf terug die de gevonden \textit{resources} beschrijft .
\end{itemize}

\subsection{Conclusie}
SPARQL is een query-taal die gebruikt wordt om gegevens op te halen van het Web. Om een correcte implementatie van SPARQL te maken moet meerdere functionaliteiten voorzien worden. Hierboven zijn slechts een (relatief klein) deel van de functionaliteiten beschreven om een beeld te schetsen van de mogelijkheden met SPARQL. Ook de beschreven delen zijn slechts zeer beperkt uitgelegd, om een minimaal beeld te schetsen van de omvang. Het belang van de werking van SPARQL is dat GeoSPARQL een uitbreiding hierop is, dus is er een correcte implementatie nodig van SPARQL alvorens GeoSPARQL geïmplementeerd kan worden. Voor de volledige en uitgebreide uitleg van SPARQL te lezen wordt best doorverwezen naar de officiële documentatie\footnote{https://www.w3.org/TR/sparql11-query/}.
\newpage
\section{Comunica}
\label{sec:comunica}

Comunica is een modulaire SPARQL \textit{query engine} voor het web, gemaakt door het IDLab van de universiteit Gent. Comunica is volledig open source (te vinden op github) en is beschikbaar via de npm \textit{package manager} \cite{taelman2018comunica}.

\subsection{Waarom Comunica?}
Comunica verschilt van de bestaande \textit{query processors} op verschillende niveau's. 

\subsubsection{Modulariteit}
Dankzij de hoge modulariteit van de Comunica \textit{query engine}, is het mogelijk om uitbreidingen en aanpassingen te doen op de algoritmes en functionaliteiten. Zo kan de gebruiker een op maat gemaakte \textit{engine} maken door de benodigde modules aan elkaar te koppelen aan de hand van een RDF configuratie bestand. Door dit document te publiceren kunnen experimenten ogenblikkelijk gereproduceerd worden door anderen \cite{taelman2018comunica}.

\subsubsection{Heterogene interfaces}
Heterogene interfaces binnen Comunica zorgen ervoor dat het mogelijk is om gefedereerd te queryen over heterogene bronnen. Zo wordt het bijvoorbeeld mogelijk om queries over eender welke combinatie van SPARQL \textit{endpoints}, \acrshort{tpf} \textit{interfaces}, \textit{datadumps} of andere types van \textit{interfaces} te evalueren \cite{taelman2018comunica}.

\subsubsection{Web gebaseerde technologieën}
Comunica is geïmplementeerd in JavaScript (of meer specifiek TypeScript) met behulp van web gebaseerde technologieën, specifieker is het geïmplementeerd als een collectie van \textit{Node modules}. Hierbovenop heeft Comunica een \textit{test coverage} van 100\% in alle modules. Hierdoor is het mogelijk om Comunica te gebruiken in zowel browsers, de \textit{command line}, het SPARQL protocol als in gelijk welke web of JavaScript applicatie \cite{taelman2018comunica}.


\subsection{Design patterns}
Om het modulaire ontwerp van Comunica mogelijk te maken, zijn er verschillende ontwerp patronen gebruikt. De drie belangrijkste zullen hieronder kort besproken worden.

\subsubsection{Publish-subscribe pattern}
Het ``publish-subscribe'' patroon werkt aan de hand van messages tussen de \textit{publishers} en de \textit{subscribers}. Dit patroon doet sterk denken aan ``observer'', waarbij alle observerende entiteiten een bericht krijgen wanneer iets verandert is van het \textit{subject} waar ze naar luisteren. Bij ``publish-subscribe'' zullen de \textit{publishers} echter de berichten vrijgeven naar bepaalde categorieën. De \textit{subscribers} kunnen hun dan inschrijven voor deze categorieën, waardoor ze deze gepubliceerde berichten kunnen ontvangen zonder hierbij kennis te hebben over de \textit{publishers}. Het grote verschil is dan ook onmiddelijk de reden waarom er bij Comunica gekozen is voor ``publish-subscribe''. ``publish-subscribe'' zorgt voor extra ontkoppeling tussen de verschillende software componenten waarbij er enkel kennis van de categorieën nodig is. In Comunica wordt dit patroon gebruikt om verschillende implementaties toe te staan voor bepaalde taken \cite{taelman2018comunica}.

\subsubsection{Actor model}
Het ``actor'' model is ontwikkeld met als doel het bekomen van zeer parallelle systemen die bestaan uit verschillende onafhankelijke agents die onderling communiceren aan de hand van \textit{messaging}. Dit is dus gelijkaardig aan het ``publish-subscribe'' patroon. Hierbij is een \textit{actor} een computationele eenheid die een specifieke taak uitvoert die reageert op berichten en die berichten kan sturen naar andere \textit{actors}. Het voordeel hiervan is dat \textit{actors} onafhankelijk van elkaar kunnen gemaakt worden om een specifieke taak te voltooien en dat deze asynchroon afgehandeld kan worden. Zo gebruikt Comunica ook het ``actor'' model om te werken naar de hoge modulariteit. De combinatie met het ``publish-subscribe'' patroon zorgt ervoor dat elke implementatie van een bepaalde taak hoort bij een aparte \textit{actor} \cite{taelman2018comunica}.

\subsubsection{Mediator pattern}
Het ``mediator'' patroon zorgt voor een verdere reductie van de koppeling tussen software componenten die met elkaar interageren. Dankzij het ``mediator'' patroon is het ook makkelijk om de interactie tussen deze componenten aan te passen. Dit is mogelijk door het inkapselen van de interactie tussen de software componenten in een \textit{mediator} component. De software componenten zullen nu, in plaats van met elkaar de interageren, communiceren door de \textit{mediator}. Zo hebben deze componenten a priori geen weet meer nodig van elkaar. Verschillende implementaties van deze \textit{mediators} zorgen voor verschillende interactie resultaten. Binnen Comunica wordt dit patroon gebruikt wanneer verschillende \textit{actors} dezelfde taak kunnen oplossen, om te beslissen welke \textit{actor} het meest geschikt is voor een taak. Eventueel kan er zelfs gekozen worden om resultaten van \textit{actors} te combineren tot één oplossing \cite{taelman2018comunica}.


\subsection{Architectuur}
Het is belangrijk om te weten dat er geen vaste ``Comunica \textit{engine}'' bestaat. Comunica is namelijk een \textit{meta engine} die geïnstantieerd kan worden in verschillende \textit{engines} gebaseerd op verschillende configuraties. Deze aanpasbaarheid wordt gerealiseerd \textit{at design-time}, gebruik makend van \textit{dependency injection} \cite{taelman2018comunica}. 

Daarnaast is er ook een enorme flexibiliteit \textit{at run-time}. Dankzij de ``publish-subscribe'', ``actor'' en ``mediator'' patronen kunnen de componenten met elkaar interageren. Dit uit zich in het ``Actor-Mediator-Bus'' patroon dat gebruikt wordt in Comunica. Dit patroon is te zien in \figureref{fig:actor-mediator-bus}. Hierin is te zien hoe een \textit{actor} een actie moet ondernemen. Deze stuurt hij vervolgens door naar de \textit{mediator}. Deze \textit{mediator} zal vervolgens een testactie sturen naar de \textit{bus}. Deze bus zal deze testactie doorsturen naar alle \textit{actors} die geregistreerd staan bij deze \textit{bus}. De \textit{bus} zal dan alle resultaten van deze testactie terugsturen naar de \textit{mediator}, waarop deze zal beslissen welke \textit{actor} het meest geschikt is voor de actie. De uiteingelijk gekozen \textit{actor} zal dan tenslotte de actie uitvoeren, zodat de \textit{mediator} het finale resultaat terug kan sturen naar de \textit{actor} die dit gehele proces heeft gestart \cite{taelman2018comunica}.

\begin{figure}
    \centering
    \includegraphics[width=\linewidth]{images/comunica-actor-mediator-bus.png}
    \caption{Actor-Mediator-Bus patroon, foto van ``Comunica: a Modular SPARQL Query Engine for the Web'' \cite{taelman2018comunica}.}
    \label{fig:actor-mediator-bus}
\end{figure}



\subsection{Conclusie}
Om tot een besluit te komen over Comunica kunnen er enkele feiten vastgesteld worden. Eerst en vooral kan er veel uitleg gegeven worden, maar is er gepoogd om de werking ervan duidelijk te maken op een korte, doch krachtige manier. Om de volledige en uitgebreide werking en analyse van Comunica te lezen kan best doorverwezen worden naar de officiële paper: ``Comunica: a Modular SPARQL Query Engine for the Web''. De belangrijkste punten om te onthouden zijn de vijf doelen die voor ogen gesteld werden bij het maken van Comunica:
\begin{enumerate}
    \item \textbf{\textit{SPARQL query evaluation}}: het moet mogelijk zijn om SPARQL queries op een correcte manier te interpreteren en een resultaat weer te geven.
    \item \textbf{Modulair}: nieuwe functionaliteiten (of bestaande functionaliteiten aanpassen) zou slechts een minimale aanpassing aan bestaande code mogen vereisen. Hierbij kan de gebruiker zelf kiezen welke modules hij wil inpluggen voor zijn persoonlijke \textit{engine}.
    \item \textbf{Heterogene \textit{interfaces}}: de mogelijkheid om te queryen naar verschillende soorten bronnen (zoals \acrshort{tpf} \textit{interfaces}, SPARQL \textit{endpoints} en data dums in RDF) moet mogelijk zijn.
    \item \textbf{\textit{Federation}}: het moet mogelijk zijn om gefedereerd te queryen. Dit betekent queryen naar verschillende bronnen. In samenhang met de heterogene \textit{interfaces} betekent dit dus queryen naar verschillende bronnen die mogelijks een verschillende soort \textit{interface} hebben.
    \item \textbf{Web gebaseerd}: Comunica is gemaakt met web technologieën zoals Javascript en RDF configuratie bestanden. Hierdoor kan Comunica werken in omgevingen zoals \textit{web browsers}, \textit{local} en zelfs in de \textit{command-line interface}.
\end{enumerate}
\newpage
\section{OGC}
\label{sec:ogc}

Alle hiervoor beschreven informatie gaat over bestaande technieken waarop verder gewerkt wordt. Vanaf nu gaat het meer over het geospatiale. Dit zijn technieken die gerespecteerd en toegepast moeten worden om zo het uiteindelijke onderzoek te kunnen doen. 

OGC staat voor Open Geospatial Consortium. Dit is een wereldwijde community die zich inzet voor het verbeteren van de manier hoe omgegaan wordt met geospatiale locatie informatie. Het OGC maakt standaarden om geospatiale informatie beschikbaar te stellen, zodat deze door gebruikers op een optimale en zo uniform mogelijke manier bereikt kan worden \cite{ogcdocs}.

Het OGC voorziet vele standaarden, daarom zullen enkel de relevante besproken worden.

\subsection{WKT}
WKT staat voor \textit{Well-known text}. Dit is een opmaak taal voor het representeren van geometrie objecten op een map. Hierbij worden de coordinaten van een positie gescheiden door een spatie (eerst de x coördinaat, daarna de y coördinaat), opeenvolgende posities binnen één structuur worden gescheiden door een komma \cite{ogcdocs}. 

De primitieve geometrieën zijn ``\textit{Point}'', ``\textit{LineString}'' en ``\textit{Polygon}''. Een ``\textit{Point}'' staat voor één exacte locatie op een map. Een ``\textit{LineString}'' gaat dan weer over een lijn, dit zijn dus de verbindingen tussen verschillende punten (bijvoorbeeld van punt1 naar punt2 en van punt2 naar punt3). Ten slotte is er een ``\textit{Polygon}'', wat staat voor een vlak. Hierbij heeft het OGC nog enkele andere voorwaarden opgesteld. Een ``\textit{Polygon}'' moet topologisch gesloten zijn, wat betekent dat het laatste punt hetzelfde moet zijn als het eerste. Daar bovenop kan een ``\textit{Polygon}'' bestaan uit een buitenste en binnenste lineaire ring. De buitenste ring slaat op het vlak, terwijl de binnenste ring slaat op een vlak dat uit het grotere vlak gehaald wordt. Hierbij stelt het OGC dat de locaties van de buitenste ring in tegenwijzerszin gegeven moeten zijn, terwijl deze van de binnenste ring in wijzerszin moeten zijn \cite{ogcdocs}.

Een verduidelijking van primitieve geometrieën is te zien in \tableref{tab:wkt_primitives}. Hierin is het eerste punt van een figuur steeds opgevuld voor verduidelijking. Bij een ``\textit{Polygon}'' met twee ringen wordt ook direct duidelijk waarom de dubbele haken er staan.

\begin{table}[ht]
\centering
\begin{tabular}{ |l|c|p{8cm}| } 
 \hline
 \rowcolor{TableHeaderColor} Type & \multicolumn{2}{c|}{Example} \\ \hline
 
 \rowcolor{TableColor} \multirow{4.5}{*}{Point} & \raisebox{-\height+2mm}[0pt][20mm]{\includegraphics[width=20mm, height=20mm]{images/wkt_point.png}} & POINT (30 10) \\ \hline
 
 \rowcolor{TableColor} \multirow{4.5}{*}{LineString} & \raisebox{-\height+2mm}[0pt][20mm]{\includegraphics[width=20mm, height=20mm]{images/wkt_line.png}} & LINESTRING (30 10, 10 20, 40 30) \\ \hline
 
 \rowcolor{TableColor} & \raisebox{-\height+2mm}[0pt][20mm]{\includegraphics[width=20mm, height=20mm]{images/wkt_polygon1.png}} & POLYGON ((40 10, 40 40, 10 40, 10 10, 40 10)) \\ \cline{2-3}
 
 \rowcolor{TableColor} \multirow{-2}{*}{Polygon} & \raisebox{-\height+2mm}[0pt][20mm]{\includegraphics[width=20mm, height=20mm]{images/wkt_polygon2.png}} & POLYGON ((40 10, 40 40, 10 40, 10 10, 40 10), (20 30, 30 30, 30 20, 20 20, 20 30) \\ \hline
\end{tabular}
\caption{Primitieve geometrieën.}
\label{tab:wkt_primitives}
\end{table}

Naast de primitieve geometrieën zijn er ook de meerdelige geometrieën. Hierin bestaan ``\textit{MultiPoint}'', ``\textit{MultiLineString}'' en ``\textit{MultiPolygon}''. Deze staan respectievelijk voor één of meer van de overeenkomstige primitieve geometrieën. Ter verduidelijking van de meerdelige geometrieën is er ook een klein voorbeeld voorzien in \tableref{tab:wkt_multipart}. Hierbij gelden uiteraard dezelfde conventies als bij de primitieve geometrieën.


\begin{table}[ht]
\centering
\begin{tabular}{ |l|c|p{8cm}| } 
 \hline
 \rowcolor{TableHeaderColor} Type & \multicolumn{2}{c|}{Example} \\ \hline
 
 \rowcolor{TableColor} \multirow{4.5}{*}{MultiPoint} & \raisebox{-\height+2mm}[0pt][20mm]{\includegraphics[width=20mm, height=20mm]{images/wkt_multipoint.png}} & MULTIPOINT ((10 40), (10 10), (20 30), (40 40)) \\ \hline
 
 \rowcolor{TableColor} \multirow{4.5}{*}{MultiLineString} & \raisebox{-\height+2mm}[0pt][20mm]{\includegraphics[width=20mm, height=20mm]{images/wkt_multiline.png}} & MULTILINESTRING ((10 10, 20 30, 10 40), (40 40, 30 20, 40 10)) \\ \hline
 
 \rowcolor{TableColor} & \raisebox{-\height+2mm}[0pt][20mm]{\includegraphics[width=20mm, height=20mm]{images/wkt_multipolygon1.png}} & MULTIPOLYGON (((10 10, 20 10, 20 40, 10 40, 10 10)), ((35 40, 25 10, 45 10, 35 40))) \\ \cline{2-3}
 
 \rowcolor{TableColor} \multirow{-2}{*}{MultiPolygon} & \raisebox{-\height+2mm}[0pt][20mm]{\includegraphics[width=20mm, height=20mm]{images/wkt_multipolygon2.png}} & MULTIPOLYGON (((10 10, 20 10, 20 40, 10 40, 10 10)), ((35 40, 25 10, 45 10, 35 40), (30 15, 30 20, 40 20, 40 15, 30 15))) \\ \hline
\end{tabular}
\caption{Meerdelige geometrieën.}
\label{tab:wkt_multipart}
\end{table}

\subsection{GML}
GML staat voor \textit{Geography Markup Language}. GML is een XML gramair die gedefinieerd is door het OGC met als doel om geospatiale informatie uit te drukken. GML doet dus hetzelfde als WKT, maar met andere notaties. Het is wel opmerkelijk dat het bij GML enkel mogelijk is om primitieve geometrieën te beschrijven, dus geen meerdelige geometrieën. Aangezien deze standaard minder vaak gebruikt wordt, zal deze ook minder gedetailleerd besproken worden. Zo zijn er drie verschillende manieren om coördinaten weer te geven in GML \cite{ogcdocs}:
\begin{itemize}
    \item ``<gml:coordinates>'': bij deze \textit{tag} worden de coördinaten gescheiden door een komma. Indien er meerdere locaties na elkaar komen worden deze dan weer gescheiden door een spatie.
    \item ``<gml:pos>'': bij deze \textit{tag} worden de coördinaten gescheiden door een spatie. Hier is het niet mogelijk om meerdere locaties na elkaar te laten komen.
    \item ``<gml:posList>'': deze \textit{tag} is gelijkaardig aan de ``pos'' \textit{tag}, maar hier is het wel mogelijk om meerdere locaties na elkaar te laten komen. Deze worden dan opnieuw gescheiden door een spatie.
\end{itemize}

Een kort voorbeeld van de ``posList'' notatie is te zien in \listingref{listing:gml}. Hierin wordt het voorbeeld van ``\textit{LineString}'' uit \tableref{tab:wkt_primitives} herschreven naar de GML notatie. Hierbij van het ``srsDimension'' attribuut op. Dit geeft aan hoeveel dimensies het punt heeft.

\begin{listing}[ht]
\begin{minted}{xml}
<gml:LineString gml:id="p21"
    srsName="http://www.opengis.net/def/crs/EPSG/0/4326">
    <gml:posList srsDimension="2">30 10 10 20 40 30</gml:posList>
</gml:LineString >
\end{minted}
\caption{Voorbeeld GML bij LineString.}
\label{listing:gml}
\end{listing}

\subsection{GeoSPARQL}
Ook GeoSPARQL is een standaard van het OGC. GeoSPARQL is gemaakt voor het representeren en queryen van geospatiale data op het semantische web. Zo definieerd GeoSPARQL een vocabulair voor het representeren van geospatiale gegevens in RDF. Verder is het belangrijk te weten dat GeoSPARQL een uitbreiding is op de SPARQL query taal, met als doel het verwerken van geospatiale gegevens. GeoSPARQL is gemaakt om zowel systemen die gebaseerd zijn op kwalitatieve spatiale redenering als systemen die gebaseerd zijn op kwantitatieve spatiale berekeningen te huisvesten. Kortom zal GeoSPARQL nieuwe filter functies definiëren voor \gls{gisdef} (\acrshort{gis}) queries, gebruik makend van standaarden die gedefinieerd zijn door het OGC. Dit is echter een zeer korte en beperkte uitleg over GeoSPARQL, de gedetailleerde specificatie worden toegelicht in \sectionref{sec:geosparql}.
\newpage
\section{GeoSPARQL}
\label{sec:geosparql}

Zoals eerder vermeld is GeoSPARQL één van de vele OGC standaarden. Specifieker is GeoSPARQL uitermate geschikt voor het uitvoeren van \acrfull{gis} queries, maar dit is zeker niet de enige mogelijkheid. Zoals eerder vermeld brengt GeoSPARQL een vocabulair voor de representatie van geospatiale gegevens, gebruik makend van RDF en is GeoSPARQL een uitbreiding op de SPARQL query taal. Het doel van GeoSPARQL is dan ook om geospatiale gegevens te verwerken. Voor het bekomen van een correcte implementatie van GeoSPARQL, zijn er normaliter meerdere voorwaarden waaraan de implementatie moet voldoen. De meeste van deze vereisten maken gebruik van de ``geo'', ``geof'' of ``geor'' ontologie \cite{ogcdocs}.

\subsection{Vereisten}
De vereisten zijn opgelijst in de officiële documentatie. Een samenvatting van het geheel is te vinden hieronder \cite{ogcdocs}. 
\begin{enumerate}
    \item \textbf{SPARQL}: Deze vereiste is meteen ook de belangrijkste. Deze stelt dat er een werkende implementatie van SPARQL aanwezig moet zijn. Dit betekent dat een implementatie GeoSPARQL ook alle mogelijkheden van SPARQL moet voorzien.
    \item \textbf{Klassen}: een correcte implementatie van GeoSPARQL moet de RDFS klassen ``geo: SpatialObject'', ``geo:Feature'' en ``geo:Geometry'' toestaan.
    \item \textbf{Properties}: verder moet een implementatie van GeoSPARQL ook verschillende attributen, zoals ``geo:sfContains'', ``geo:hasGeometry'' en ``geo:dimension'' naast vele andere, toestaan.
    \item \textbf{WKT}: RDFS \textit{literals} van het type ``geo:wktLiteral'' bevatten mogelijks een URI die het coördinaat stelsel van deze coördinaten beschrijft. Indien deze URI niet aanwezig is wordt er gekozen voor ``<http://www.opengis.net/def/crs/OGC/1.3/CRS84>''. Daarnaast moeten de coördinaten geïnterpreteerd worden zoals beschreven in het referentiesysteem.
    \item \textbf{GML}: naast WKT moet GML ook ondersteund worden. De implementatie moet zelf documenteren welke profielen ondersteund worden.
    \item \textbf{Functies}: Bij de implementatie van GeoSPARQL moeten meerdere functies ondersteund worden. Hierin is onderscheid gemaakt tussen topologische functies (zoals onder andere ``geof:sfContains'' en ``geof:ehContains'') en niet-topologische functies (zoals ``geof:distance'' en ``geof:union'').
    \item \textbf{Entailment}: GeoSPARQL moet dezelfde semantiek voor \textit{basic graph pattern matching} hanteren als SPARQL. Dit houdt in dat een functie als predikaat moet kunnen omgezet worden naar een query die ook functies berekent om aan een correct resultaat te komen. Hiervoor gebruikt het regels zoals onder andere ``geor:sfContains''.
\end{enumerate}

\subsection{Architectuur}
\label{subsec:geosparql_architecture}
\begin{figure}[ht]
    \centering
    \includegraphics[width=0.7\linewidth]{images/geosparql_architecture.png}
    \caption{Vereenvoudigd diagram van de GeoSPARQL klassen ``Feature'' en ``Geometry'' met sommige properties (figuur gebaseerd op \cite{geosparqlsupport}).}
    \label{fig:geosparql_architecture}
\end{figure}

Om de architectuur vereenvoudigd weer te geven kan \figureref{fig:geosparql_architecture} helpen. Hierin is te zien dat er één hoofdklasse is waar de andere van overerven, genaamd ``SpatialObject''. Het belangrijke hier is het verschil tussen een ``Feature'' en een ``Geometry''. Een ``Geometry'' is het effectieve spatiale object, dat weergegeven kan worden als WKT of als GML (zie verder in \subsectionref{subsec:geosparql_properties}). Eén van de belangrijke (en niet vanzelfsprekende) ontwerpkeuzes is het gebruik van de ``Feature'' klasse. Deze werkt als een soort wrapper klasse voor ``Geometry''. Dit is belangrijk om te weten voor verderop in \subsectionref{subsec:geosparql_rewrite_query}, maar daar zal nogmaals terug verwezen worden naar deze subsectie.

\subsection{Properties}
\label{subsec:geosparql_properties}
GeoSPARQL voorzien een hele lijst met properties die voorzien moet worden. Vanwege de eerder korte periode om een volledige implementatie van GeoSPARQL te maken, wordt hier minder aandacht aan gegeven. Enkele voorbeelden van deze attributen zijn de volgende \cite{ogcdocs}:

\begin{itemize}
    \item \textbf{geo:isEmpty}: dit attribuut zal ``true'' terug geven indien de geometrie geen punten bevat.
    \item \textbf{geo:isSimple}: Dit attribuut zal ``true'' geven indien de geometrie geen intersecties met zichzelf bevat. Dit kan wel uitgezonderd zijn \textit{boundary} zijn.
    \item \textbf{geo:hasSerialization}: Dit attribuut wordt gebruikt om een geometrie te connecteren met zijn text-gebaseerde serialisatie. Dit kan WKT of GML zijn, maar aangezien deze uitgebreid besproken zijn in \sectionref{sec:ogc}, wordt dit hier niet herhaald. Het kan wel nogmaals benadrukt worden dat deze \textit{literal} optioneel kan bijhouden welk topologisch referentiesysteem gebruikt wordt. Het is wel belangrijk te weten waarom deze referentiesystemen zo belangrijk zijn. Australië bijvoorbeeld, verdrijft jaarlijks wat van zijn oorspronkelijke locatie. Na een aantal jaar zou elk huis op de kaart een volledig huis verder liggen, waardoor de informatie nutteloos geworden is. GeoSPARQL lost dit op met het gebruik van de referentiesystemen.
\end{itemize}


\subsection{Topologische relaties}
\label{subsec:topologische_relaties}
Om topologische relaties te kunnen beschrijven, wordt er gebruik gemaakt van drie relatiefamilies. Deze relatiefamilies beschrijven grotendeels gelijkaardige topologische relaties, maar met licht verschillende specificaties. De drie relatiefamilies zijn ``Simple Features (sf)'', ``Egenhofer (eh)'' en ``RCC8''. Om de spatiale relaties te beschrijven wordt gebruik gemaakt van een ``DE-9IM'' patroon. Voordat de relatiefamilies uitgelegd kunnen worden, is een volledig begrip van de DE-9IM notatie noodzakelijk.

\subsubsection{DE-9IM}
\begin{figure}[ht]
    \centering
    \includegraphics[width=0.9\linewidth]{images/spatial_objects_DE-9IM.png}
    \caption{Spatiale objecten met hun \textit{interior}, \textit{boundary} en \textit{exterior}: (a) Een punt; (b) Een lijn; (c) Een vlak. Figuur van \cite{shen2018classification}}.
    \label{fig:de-9im}
\end{figure}

DE-9IM staat voor \textit{Dimensionally Extended Nine-Intersection Model}. Dit model kan afgebeeld worden als een 3x3 matrix, waarbij (in deze volgorde) de \textit{interior}, \textit{boundary} en \textit{exterior} van het ene spatiale object vergeleken wordt met deze van het andere spatiale object. Deze matrix geeft de dimensies van de intersectie van deze objecten weer. De betekenis van \textit{interior}, \textit{boundary} en \textit{exterior} voor de verschillende soorten spatiale objecten is verduidelijkt in \figureref{fig:de-9im}. De volledige 3x3 matrix voor de objecten a en b is van de volgende vorm \cite{shen2018classification}:

\begin{equation*}
    DE-9IM(a,b) = 
    \begin{bmatrix}
    dim(I(a)\cap I(b)) & dim(I(a)\cap B(b)) & dim(I(a)\cap E(b))\\
    dim(B(a)\cap I(b)) & dim(B(a)\cap B(b)) & dim(B(a)\cap E(b))\\
    dim(E(a)\cap I(b)) & dim(E(a)\cap B(b)) & dim(E(a)\cap E(b))
    \end{bmatrix}
\end{equation*}

\begin{figure}[ht]
    \centering
    \includegraphics[width=0.5\linewidth]{images/de-9im_example.png}
    \caption{Voorbeeld voor de DE-9IM matrix.}
    \label{fig:de-9im_example}
\end{figure}

Wanneer dit toegepast wordt op de objecten die te zien zijn in \figureref{fig:de-9im_example}, dan wordt de volgende matrix bekomen:

\begin{equation*}
    DE9IM(a,b) = 
    \begin{bmatrix}
    2 & 1 & 2\\
    1 & 0 & 1\\
    2 & 1 & 2
    \end{bmatrix}
\end{equation*}

Voor het bepalen van de relaties van de GeoSPARQL families zijn er echter nog iets andere regels opgesteld. Hier heeft het OGC nieuwe betekenissen geïntroduceerd: 
\begin{enumerate}
    \item \textbf{Empty}: dit wordt genoteerd als -1.
    \item \textbf{True}: Dit geldt wanneer de waarden 0, 1, 2 voorkomen. Bij deze waarde kan soms gespecifieerd worden dat het een bepaalde waarde exact moet uitkomen. Het kan ook genoteerd worden als ``T'' (dit is trouwens meer wel dan niet het geval).
    \item \textbf{False}: dit geldt wanneer de waarde -1 wordt uitgekomen, maar wordt genoteerd als ``F''.
    \item \textbf{Don't care}: Dit betekent dat eender welke waarde mag voorkomen en dat dit hier niet naar gekeken moet worden. Dit wordt genoteerd als ``*''.
\end{enumerate}
Hierbij wordt het patroon geschreven als een sequentie van negen karakters. Als deze van links naar rechts gelezen wordt, dan stelt dit de DE-9IM matrix voor van linksboven beginnend, rij per rij opvullend. Het volgende patroon staat dus voor de overeenkomstige matrix:

\begin{equation*}
    T**T**T** = 
    \begin{bmatrix}
    T & * & *\\
    T & * & *\\
    T & * & *
    \end{bmatrix}
\end{equation*}
Indien er meer dan één lijn staat, wil dit zeggen dat één van de mogelijke matrices voldoende is om te besluiten dat de relatie klopt.

Ten slotte kunnen ook niet alle relaties voorkomen bij elke object-combinatie (bijvoorbeeld twee punten kunnen elkaar niet kruisen). Daarom wordt er ook een nieuwe notatie toegevoegd in de GeoSPARQL-documentatie, om te vermelden op welke spatiale objecten het DE-9IM intersectie patroon van toepassing is. Hierbij staat symbool ``P'' voor 0-dimensionale geometrieën (zoals punten). Het symbool ``L'' staat dan weer voor 1-dimensionale geometrieën (zoals lijnen). Het laatste symbool is ``A'', wat dan weer staat voor 2-dimensionale geometrieën (zoals vlakken) \cite{ogcdocs}.


\subsubsection{Simple Features}
\label{subsubsec:simple_features}
De eerste relatiefamilie is de ``Simple Feature'' family. De regels van deze familie worden weergegeven in \tableref{tab:topo_sf}. Hierin is exact te zien hoe de implementaties verwacht worden te werken. Om de grootte van deze masterproef in te perken is enkel deze relatiefamilie uitgewerkt en wordt dus ook enkel deze relatiefamilie besproken. Indien het voorbeeld van de ``contains'' functie besproken wordt, moet dit als volgt geïnterpreteerd worden: ``a contains b is waar, indien de intersectie van hun \textit{interiors} bestaat en hierbovenop de intersectie van de \textit{exterior} van a met zowel de \textit{interior} als de \textit{boundary} van b niet bestaat''. Dit betekent dus dat b in a ligt en dat geen enkel deel van b buiten a ligt \cite{ogcdocs}.


\begin{table}[ht]
    \centering
    \begin{tabular}{ |p{1.5cm}|l|l|p{2cm}|p{2.5cm}| } 
        \hline
        \rowcolor{TableHeaderColor} Relation Name & Relation URI & Domain/Range & Applies To Geometry Types & DE-9IM Intersection pattern \\ \hline
        
        \rowcolor{TableColor} equals & geo:sfEquals & geo:SpatialObject & All & (TFFFTFFFT) \\ \hline
        
        \rowcolor{TableColor} disjoint & geo:sfDisjoint & geo:SpatialObject & All & (FF*FF****) \\ \hline
        
        \rowcolor{TableColor} intersects & geo:sfIntersects & geo:SpatialObject & All & (T********
        *T*******
        ***T*****
        ****T****)  \\ \hline
        
        \rowcolor{TableColor} touches & geo:sfTouches & geo:SpatialObject & All except P/P & (FT*******
        F**T*****
        F***T****) \\ \hline
        
        \rowcolor{TableColor} within & geo:sfWithin & geo:SpatialObject & All & (T*F**F***) \\ \hline
        
        \rowcolor{TableColor} contains & geo:sfContains & geo:SpatialObject & All & (T*****FF*) \\ \hline
        
        \rowcolor{TableColor} overlaps & geo:sfOverlaps & geo:SpatialObject & A/A, P/P, L/L & ((T*T***T**)
        for A/A, P/P;
        (1*T***T**)
        for L/L) \\ \hline
        
        \rowcolor{TableColor} crosses & geo:sfCrosses & geo:SpatialObject & P/L, P/A, L/A, L/L & ((T*T***T**)
        for P/L, P/A,
         L/A;
        (0********)
        for L/L) \\ \hline
        
    \end{tabular}
    \caption{Simple Features topologische relaties (tabel van \cite{ogcdocs}).}
    \label{tab:topo_sf}
\end{table}


\subsection{Niet-topologische relaties}

Verder zijn er ook de niet-topologische relaties. Verschillende van deze functies gebruiken een meeteenheid URI. Daarom heeft het OGC enkele standaard meeteenheden gedefinieerd, te vinden onder de \textit{namespace} ``http://www.opengis.net/def/uom/OGC/''. Een voorbeeld van zo een meeteenheid is ``<http://www.opengis.net/def/uom/OGC/metre>''. Enkele van de niet-topologische functies zijn de volgende: \cite{ogcdocs}.

\begin{itemize}
    \item \textbf{geof:distance}: deze functie moet de kortst mogelijke afstand geven tussen twee objecten.
    \item \textbf{geof:buffer}: deze functie geeft een geometrie terug die alle punten bevat die binnen een bepaalde radius van een meegegeven geometrie liggen.
    \item \textbf{geof:convexHull}: deze functie geeft een geometrie terug die het kleinste convexe omhulsel is dat alle punten omvat van een meegegeven geometrie.
    \item \textbf{geof:intersection}: deze functie geeft een geometrie terug die staat voor alle punten van de intersectie tussen twee geometrieën.
    \item \textbf{geof:union}: deze functie geeft een geometrie terug die staat voor alle punten in de unie van twee geometrieën.
    \item \textbf{geof:envelope}: Deze functie geeft de minimaal omvattende (\textit{bounding}) \textit{box} weer voor een geometrische figuur. Dit betekent dus de kleinst mogelijke rechthoek waar elk punt van de geometrie in zit.
    
\end{itemize}

Ten slotte is het belangrijk te vermelden dat alle berekeningen horen te gebeuren in het referentiesysteem van de eerste geometrie die meegegeven is aan een functie, zowel bij topologische als niet-topologische functies \cite{ogcdocs}.


\subsection{Query Rewrite Extension}
\label{subsec:geosparql_rewrite_query}
De laatste uitdaging is het probleem waar de topologische functies als predikaat staan. Hierbij zou verwacht worden dat deze op voorhand berekend zijn en zo opgeslagen zijn in de RDF dataset. Hierbij is echter het probleem dat wanneer dit niet op voorhand berekend is, men nog steeds de correcte oplossing wil. Wanneer een deel van de gegevens uit de ene dataset komt en het andere deel uit een andere bron, zelfs dan wordt een correct antwoord verwacht, hoewel hier geen optie is tot het op voorhand uitrekenen \cite{ogcdocs}. 

Hiervoor wordt gebruik gemaakt van de \textit{query rewrite} uitbreiding. Deze techniek zal de query uitbreiden, door de unie (de SPARQL \textit{union}, niet de niet-topologische functie van GeoSPARQL!) te nemen van de oorspronkelijke stelling met enerzijds de relatie als predikaat en anderzijds het nieuwe uitgebreide deel waarbij dezelfde relatie uitgerekend wordt als functie. Hierbij is het ook belangrijk rekening te houden met het verschil tussen een ``Feature'' en een ``Geometry'' (zoals uitgelegd in \subsectionref{subsec:geosparql_architecture}), waardoor er vier extra delen nodig zijn voor de mogelijke combinaties tussen ``Feature'' en ``Geometry'' \cite{ogcdocs}. 

Zo heeft het OGC een template gemaakt om dit te herschrijven. Bij deze template zijn er een aantal placeholders gebruikt \cite{ogcdocs}:
\begin{itemize}
    \item \textbf{ogc:relation}: deze placeholder staat voor de relatie die gebruikt wordt (dit zou bijvoorbeeld de ``geo:sfContains'' kunnen zijn).
    \item \textbf{ogc:function}: deze placeholder staat voor de overeenkomstige functie bij de relatie die gebruikt wordt (in het voorbeeld van ``geo:sfContains'' zou de functie ``geof:sfContains'' zijn).
    \item \textbf{ogc:asGeomLiteral}: deze placeholder staat voor één van de serialisatie technieken om het object te bekomen.
\end{itemize}

Een voorbeeld voor het herschrijven van een query is te vinden in \listingref{listing:geosparql_query_to_rewrite}, maar de uiteindelijke template voor het herschrijven van de query is te vinden in \listingref{listing:geosparql_rewrite_query}.

\begin{listing}[ht]
    \begin{minted}{sparql}
        select *
        where {
            { ?f1 ogc:relation ?f2 . }
        }
    \end{minted}
    \caption{Voorbeeld query om te herschrijven.}
    \label{listing:geosparql_query_to_rewrite}
\end{listing}

\begin{listing}[ht]
    \begin{minted}{sparql}
        select *
        where {
            { ?f1 ogc:relation ?f2 . }
            UNION
            # feature - feature rule
            {   ?f1 geo:hasDefaultGeometry ?g1 . 
                ?f2 geo:hasDefaultGeometry ?g2 .
                ?g1 ogc:asGeomLiteral ?g1Serial .
                ?g2 ogc:asGeomLiteral ?g2Serial .
                filter(ogc:function(?g1Serial, ?g2Serial)) }
            UNION
            # feature - geometry rule
            {   ?f1 geo:hasDefaultGeometry ?g1 . 
                ?g1 ogc:asGeomLiteral ?g1Serial .
                ?f2 ogc:asGeomLiteral ?g2Serial .
                filter(ogc:function(?g1Serial, ?g2Serial)) }
            UNION
            # geometry - feature rule
            {   ?f2 geo:hasDefaultGeometry ?g2 .
                ?f1 ogc:asGeomLiteral ?g1Serial .
                ?g2 ogc:asGeomLiteral ?g2Serial .
                filter(ogc:function(?g1Serial, ?g2Serial)) }
            UNION
            # geometry - geometry rule
            {   ?f1 ogc:asGeomLiteral ?g1Serial .
                ?f2 ogc:asGeomLiteral ?g2Serial .
                filter(ogc:function(?g1Serial, ?g2Serial)) }
        }
    \end{minted}
    \caption{Template om queries te herschrijven (listing van \cite{ogcdocs}).}
    \label{listing:geosparql_rewrite_query}
\end{listing}
\newpage
\chapter{Implementatie}
\label{chap:implementatie}











\todo{uitleg begonnen in comunica -> minimale moeite door modulair systeem + hierbij hebben we al een werkende sparql implementatie}

\todo{uitbreiding op SPARQL, geïmplementeerd in Comunica met minimale moeite dankzij modulariteit}
\todo{uitleg zelf maken vs terraformer vs turfjs (modulair) voor geospatiale relatie}
\todo{uitleg projecties vanwege verschuivingen van de aarde (vb australie) (lambert vs andere) + uitwerking (mss proj4js) + beperkingen in huidige opengis docs (weinig projecties voorzien) + beperkingen proj4js}
\chapter{Interfaces}
\label{chap:interfaces}
Het doel van deze masterproef is het maken van een implementatie van GeoSPARQL in Comunica en te controleren of het hierbij mogelijk is om GeoSPARQL functionaliteiten te implementeren in deze \textit{query engine} die over verschillende heterogene data bronnen kan queryen. Hierbij worden de gegevens op de gebruikelijke manier opgehaald, maar worden de geospatiale relaties uitgerekend op de client-side. Hierbij moet gecontroleerd worden of dit mogelijk is bij de bronnen die momenteel door Comunica ondersteund worden. De belangrijkste bronnen zijn ``data dumps'', ``triple pattern fragment interfaces'' en ``sparql endpoints''. In de komende secties wordt besproken of dit mogelijk is en hoe dit werkt. De eerstvolgende sectie geeft echter een kort overzicht van de gebruikte dataset voor deze tests.

\section{Testset}
\label{sec:testset}
Bij de testset is ervoor gekozen om een oppervlakkige tekening te maken van België in een aangepaste schaal. Deze keuze is gemaakt vanwegen meerdere redenen. Om te beginnen laat deze aangepaste schaal zeer makkelijk toe om (als mens) vlakken te beschrijven. Aangezien dit een technisch en bovendien vooruitstrevend onderwerp is, is het handig om terug te keren naar een bekender terein. Daarom is de bekendheid van deze \textit{use case} meteen de tweede reden van deze keuze. Een derde reden is dat deze dataset eerder klein is, waardoor mogelijke fouten of onlogische oplossingen hierbij nogmaals getest worden. Ten slotte is dit ook de perfecte dataset voor het geven van demonstraties omdat deze set herkenbaar is voor het publiek. Een visualisatie van deze dataset is te zien in \figureref{fig:demoset}. De effectieve dataset is dan weer te vinden op GitHub Gist\footnote{https://gist.github.com/dreeki/e48bbe533a4b1191045b3652ff2c9c81}. Deze dataset is bovendien opgesplitst in vijf verschillende bestanden (namelijk: ``land'', ``gewest'', ``provincie'', ``weg'' en ``stad'') zodat deze dataset bruikbaar is voor het verifiëren dat gefedereerd queryen nog steeds mogelijk is. Het geheel van deze testing wordt uitgevoerd in de testomgeving die besproken werd in \sectionref{sec:testomgeving}. Hierbij zorgt de \textit{execution log} ervoor dat alles duidelijk is hoe het geheel in zijn werk gaat.

\begin{figure}
    \centering
    \includegraphics[width=\linewidth]{images/geosparql_demo.png}
    \caption{Testset voor het testen van de verschillende bronnen.}
    \label{fig:demoset}
\end{figure}


\subsection{queries}
Voor de uitvoering van de tests zijn verschillende queries opgesteld om mee te testen. In de testomgeving zelf kunnen hierbij vervolgens de bronnen aangepast worden naar de te testen bronnen. Het is echter wel nog op te merken dat de ontologieën bij de testomgeving reeds in code aangegeven werden. Bij gevolg worden deze niet opnieuw mee opgenomen in de geschreven query, maar in de achtergrond worden deze dus wel nog gebruikt.
\todo{my:hasExactGeometry aanpassen en op gist aanpassen!}


\subsubsection{Query 1}
De eerste \textit{query} zoekt naar alle provincies die binnen Vlaanderen liggen. Om dit te kunnen doen zal de \textit{query engine} eerst de vorm van Vlaanderen zelf opzoeken. Daarna zal hij de vorm van alle provincies zoeken zodat hij met de ``sfContains'' functie van GeoSPARQL uiteindelijk kan filteren. Deze query is te zien in \listingref{listing:find_provinces_flanders}.

\begin{listing}[ht]
    \begin{minted}{sparql}
        SELECT ?f
        WHERE {
            gewest:Vlaanderen my:hasExactGeometry ?aGeom .
            ?aGeom geo:asWKT ?aWKT .
            ?f a my:Provincie .
            ?f my:hasExactGeometry ?fGeom .
            ?fGeom geo:asWKT ?fWKT .
            FILTER (geof:sfContains(?aWKT, ?fWKT))
        }
    \end{minted}
    \caption{Find all the provinces in Flanders.}
    \label{listing:find_provinces_flanders}
\end{listing}


\subsubsection{Query 2}
De tweede \textit{query} is zeer gelijkaardig aan de eerste, maar hier is één groot verschil. Deze \textit{query} zoekt naar alles dat binnen Vlaanderen ligt. Aangezien ``sfContains'' functie stelt dat een geospatiaal identieke vorm steeds binnen de andere ligt, zou deze query Vlaanderen zelf ook als een oplossing zien. Aangezien dit niet het verwachte resultaat is, wordt gebruik gemaakt van de negatie van de ``sameterm'' functie van SPARQL. Dit wijst er nogmaals op dat een werkende implementatie van SPARQL een vereiste is voor het maken van een GeoSPARQL implementatie. Deze query is te zien in \listingref{listing:find_everything_flanders}.

\begin{listing}[ht]
    \begin{minted}{sparql}
        SELECT ?f
        WHERE {
            gewest:Vlaanderen my:hasExactGeometry ?aGeom .
            ?aGeom geo:asWKT ?aWKT .
            ?f my:hasExactGeometry ?fGeom .
            ?fGeom geo:asWKT ?fWKT .
            FILTER (geof:sfContains(?aWKT, ?fWKT) && !sameterm(?aWKT, ?fWKT))
        }
    \end{minted}
    \caption{Find everything that's geospatially inside Flanders.}
    \label{listing:find_everything_flanders}
\end{listing}


\subsubsection{Query 3}
De derde query gaat dan weer over het vinden van de wegen en provincies die binnen België liggen. Deze query toont nogmaals aan hoe gelijkaardig SQL en SPARQL zijn. Deze query is te zien in \listingref{listing:find_provinces_roads_belgium}.

\begin{listing}[ht]
    \begin{minted}{sparql}
        SELECT ?f
        WHERE {
            land:België my:hasExactGeometry ?aGeom .
            ?aGeom geo:asWKT ?aWKT .
            {
                ?f a my:Provincie .
            }
                UNION
            {
                ?f a my:Weg .
            }
            ?f my:hasExactGeometry ?fGeom .
            ?fGeom geo:asWKT ?fWKT .
            FILTER (geof:sfContains(?aWKT, ?fWKT))
        }
    \end{minted}
    \caption{Find all the provinces and roads in Belgium.}
    \label{listing:find_provinces_roads_belgium}
\end{listing}


\subsubsection{Query 4}
Bij de vierde query worden alle wegen die door Oost-Vlaanderen gaan opgezocht. Dit zou bijvoorbeeld handig kunnen zijn wanneer iemand de snelwegen wil vinden die makkelijk bereikbaar zijn vanuit Oost-Vlaanderen. Hierbij wordt de functie ``sfIntersects'' van GeoSPARQL gebruikt. Deze query is te zien in \listingref{listing:find_roads_passing_east_flanders}.

\begin{listing}[ht]
    \begin{minted}{sparql}
        SELECT ?f
        WHERE {
            prov:OostVlaanderen my:hasExactGeometry ?aGeom .
            ?aGeom geo:asWKT ?aWKT .
            ?f a my:Weg .
            ?f my:hasExactGeometry ?fGeom .
            ?fGeom geo:asWKT ?fWKT .
            FILTER (geof:sfIntersects(?aWKT, ?fWKT))
        }
    \end{minted}
    \caption{Find all the roads that pass through East-Flanders.}
    \label{listing:find_roads_passing_east_flanders}
\end{listing}


\subsubsection{Query 5}
De vijfde query toont aan dat het mogelijk is om manueel een vorm te voorzien om op te filteren. Deze vorm staat (bij de aangepaste schaal) voor de \textit{bounding box} van Vlaams-Brabant en Waals-Brabant. Deze query zal alles weergeven dat zich binnen deze vorm bevindt. Voor de verandering wordt hier de ``sfWithin'' functie van GeoSPARQL gebruikt, maar dit zou evengoed mogelijk zijn met de ``sfContains'' functie. Deze query is te zien in \listingref{listing:find_everything_bounding_box}.

\begin{listing}[ht]
    \begin{minted}{sparql}
        SELECT ?f
        WHERE {
            ?f my:hasExactGeometry ?fGeom .
            ?fGeom geo:asWKT ?fWKT .
            FILTER (geof:sfWithin(?fWKT, '''Polygon((7 9.25, 13.5 9.25, 13.5 13.25, 
            7 13.25, 7 9.25))'''^^geo:wktLiteral))
        }
    \end{minted}
    \caption{Find everything inside the bounding box of Brabant.}
    \label{listing:find_everything_bounding_box}
\end{listing}




\todo{misschien nog extra queries} 
\newpage
\section{Data dump}
\label{sec:data-dump}
De eerste bron om te controleren wordt gebruikt als baseline. Dit is de ``data dump''. Dit is een gewoon bestand in RDF formaat dat door de \textit{query engine} opgehaald (lees gedownload) wordt. Vervolgens wordt door de \textit{query engine} gecontroleerd of het effectief wel een bestandsbron is. Eenmaal dit voldaan is, worden steeds de kleinste patronen gezocht die voldoen aan de bron. Dit is nodig omdat de \textit{query engine} zo op de meest performante manier de \textit{joins} kan uitvoeren. Zo kan ten slotte de filter-functie de overbodige oplossingen weghalen. Deze filter-functie is voorzien door GeoSPARQL. Bovendien voorziet Comunica functionaliteiten om data-entiteiten uit de databron te extraheren, wat hier gebeurt op de client. Over deze entiteiten moet een filter functie geëvalueerd kunnen worden. Hierdoor is het mogelijk om data dumps te queryen met GeoSPARQL. Aangezien data dumps letterlijk bestanden zijn zonder een eigen voorziening van logica, is het triviaal dat dit afgehandeld moet kunnen worden. De data dump wordt daarom de \textit{baseline} van dit onderzoek, waarbij er gepoogd wordt om dezelfde resultaten bij andere bronnen ook te behalen.

Bij het testen van de data dump worden de GitHub Gist bestanden (zie \sectionref{sec:testset}) rechtstreeks gebruikt. Hier is geen enkel ander programma dat als aanspreekpunt gebruikt wordt. Hierbij is dus ook duidelijk dat het beschreven proces van hierboven correct doorlopen is. Dit is bovendien de manier van werken die gebruikt is bij het maken en controleren van de implementatie. 
\newpage
\section{Triple pattern fragment interface}
\label{sec:impl_tpf_interface}
De volgende bron is de ``triple pattern fragment interface''. Deze server staat tussen de op te vragen bestanden (dus de effectieve gegevens) en de client. Deze server zal ervoor zorgen dat het niet langer nodig is om alle data op te halen, maar in de plaats zal de server de SPARQL query splitsen in verschillende triple pattern fragment requests. Deze vragen alle triples die voldoen aan een enkel tripple pattern fragment en \textit{joinen} dan de resultaten op de client. De filtering gebeurt daarna ook op de client.

Net zoals bij de data dump (zie \sectionref{sec:data-dump}) vraagt de \textit{query engine} de bron op. Hierbij zal hij de bron identificeren als een ``qpf source'', wat staat voor ``Quad pattern fragment''. Dit is eigenlijk de \textit{triple pattern fragment} met hierbij een extra veld (graph) toegevoegd, maar hier wordt niet verder op in gegaan. Vervolgens begint de \textit{query engine} de query op de splitsen in \acrfull{tpf} queries, zodat deze \acrshort{tpf} queries geoptimaliseerd kunnen worden in een volgorde die gebaseerd is op de initiële count query. Hierna worden deze één voor één uitgevoerd. Hij zal vervolgens de kleinste patronen opvragen, zodat de correcte informatie opgevraagd kan worden, met hierbij een minimale hoeveelheid aan overbodige informatie. Dit is enkel mogelijk dankzij de \acrshort{tpf} interface. Dankzij deze manier van werken kan wederom de uiteindelijke filtering van de \textit{queries} louter op de client-side gebeuren.

Voor het opzetten van deze test is gebruik gemaakt van de ``Linked Data Fragments Server''. Deze bouwt een \acrshort{tpf} interface op boven een set van bronbestanden, waarvoor opnieuw de bestanden van GitHub Gist gebruikt zijn, net zoals bij de data dump. Op deze manier is ook verzekerd dat er met dezelfde gegevens gewerkt wordt. 

Als kleine opmerking kan nog vermeld worden dat een \acrshort{tpf} interface gebruikt wordt voor twee redenen. Ten eerste hoeft de client zo niet alle data te downloaden, maar kan de server slechts een fragment (vandaar de naam, deze komt eigenlijk van ``Linked Data Fragments'') van deze data teruggeven. De tweede reden is dat deze filtering meestal (= niet in alle gevallen) zorgt voor een verbeterde performantie. Bij het testen was dit echter niet terug te vinden. Zo blijkt het uitvoeren van de queries met de data dump sneller te gaan dan met de \acrshort{tpf} interface. Hier is echter een logische verklaring voor. De datasets die gebruikt zijn, zijn relatief gezien kleine datasets. Bovendien bevatten deze enkel de noodzakelijke gegevens, waardoor de dataset volledig nodig is voor het uitvoeren van de query. Hierdoor kan er niet genoten worden van de voordelen van de \acrshort{tpf} interface, maar wordt enkel de extra \textit{overhead} waargenomen. Dit gaat echter buiten de \textit{scope} van deze masterproef, daarom werd hier verder geen onderzoek naar gedaan, noch benchmarking van de performantie. Dit is eerder een opmerking bij de ondervindingen. 
\newpage
\section{SPARQL endpoint}
\label{sec:impl_sparql_endpoint}
De laatste bron om te testen is meteen de moeilijkste. Bij het SPARQL endpoint is het de bedoeling dat een GeoSPARQL query uitgevoerd kan worden door de gegevens op te vragen aan dit SPARQL endpoint. Comunica zelf is gemaakt om queries uit te voeren over RDF bronnen, wat het zeer handig maakt om SPARQL queries uit te voeren. Hierdoor lijkt het logisch om de query in zijn geheel door te sturen in het geval van een SPARQL endpoint als bron. Dit SPARQL endpoint is namelijk in staat om volledig automoon een antwoord te geven op de query. Dit is echter niet hoe het in zijn werk gaat. Eén van de redenen hiervoor is dat gefedereerd queryen zo niet mogelijk zou zijn wanneer er naast het SPARQL endpoint nog een andere bron zou zijn. Zo moet de samenvoegingen van de antwoorden op de client-side gebeuren. 

Bij een SPARQL endpoint zal de \textit{query engine} de verschillende RDF triples van de query overlopen. Dit doet hij in twee stappen. De eerste stap is een ``count'' zodat hij weet hoeveel RDF triples van de bron overeen komen met de RDF triples van de query. Dit wordt gedaan zodat de \textit{query engine} weet welke volgorde optimaal is om de data op te halen, zodat hij dit optimaal kan joinen. De tweede stap is het effectieve ophalen van het resultaat, waarbij hij dus alle antwoorden vraagt die voldoen aan slechts één RDF triple van de query. Wanneer dit voor alles gedaan is zoekt hij het kleinste patroon, zodat hij vervolgens de matchende RDF triples kan ophalen. Het kleinste patroon wordt gekozen om het aantal matchende resultaten te minimaliseren voor performantie redenen.

Zo wordt uiteindelijk alle benodigde informatie uit het SPARQL endpoint systematisch opgehaald, zodat de filter bij de oorspronkelijke query onafhankelijk van de bronnen kan uitgevoerd worden. Dit betekent ook dat de filter functies steeds dezelfde implementatie hebben (namelijk deze van de \textit{query engine} op de client, niet deze van de bron). Dit laat echter ook toe om te filteren met de GeoSPARQL functies. Dankzij deze werkwijze is het dus effectief mogelijk om de GeoSPARQL functionaliteit toe te passen bij het opvragen aan een SPARQL endpoint. 
\newpage
\begin{savequote}[0.55\linewidth]
	``Dream big and dare to fail.''
	\qauthor{\textasciitilde 
    Norman Vaughan}
\end{savequote}

\chapter{Conclusie}
\label{chap:conclusie}
In dit hoofdstuk worden de resultaten, besproken in \chapterref{chap:interfaces}, geïnterpreteerd. Er wordt een antwoord geformuleerd op de vragen uit \sectionref{sec:onderzoeksvraag}.

\textbf{Onderzoeksvraag: Welke ``Linked Data publicatie''-interfaces kunnen uitgebreid worden met GeoSPARQL-functionaliteiten door de filtering op de client uit te voeren?}

De masterproef brengt een oplossing om de ``Linked Data publicatie''-interfaces uit te breiden met GeoSPARQL-functionaliteiten. Dit betekent dat er gelinkte data online gepubliceerd worden. Deze data kunnen opgehaald worden met behulp van SPARQL (zie \sectionref{sec:sparql}). 

De vraag is hoe deze opvraging kan uitgebreid worden met GeoSPARQL-functionaliteiten. Hiervoor wordt gebruik gemaakt van de al bestaande implementatie van Comunica. Door Comunica uit te breiden met deze GeoSPARQL-functionaliteiten wordt gepoogd om op dezelfde manier te kunnen werken als voordien, maar ditmaal met GeoSPARQL-functionaliteiten. Specifiek hierbij worden de data opgehaald en gefilterd op de client zelf (zoals besproken in \chapterref{chap:interfaces}).

De meest voorkomende ``Linked Data publicatie''-interfaces voor het gebruik van geospatiale data zijn ``data dumps'', ``\acrshort{tpf} interfaces'' en ``SPARQL endpoints''.

\textbf{Hypothese 1: Het is mogelijk om GeoSPARQL queries uit te voeren over ``data dumps'' waarbij de filtering op de client-side gebeurt.}

Bij een ``data dump'' worden de data volledig gedownload op de client. Hier zal de client vervolgens de resultaten joinen zoals nodig in de query. Ten slotte zal de client over de volledige dataset filteren, om zo tot het correcte resultaat te komen. 

Hieruit volgt dat Hypothese 1 correct is.

\textbf{Hypothese 2: Het is mogelijk om GeoSPARQL queries uit te voeren over ``\acrshort{tpf} interfaces'' door de filtering op de client-side uit te voeren.}

De ``\acrshort{tpf} interface'' is een server die een dataset bevat. Deze server bevat functionaliteiten om te kunnen antwoorden op vragen naar een \textit{triple pattern fragment}. Zo wordt de volledige query opgesplitst, zodat de ``\acrshort{tpf} interface'' het kleinst mogelijke deel van de dataset (dat alle vereiste data bevat) kan terug geven. Hierbij zal de client wederom de resultaten joinen om hierop te kunnen filteren zodat het correcte resultaat bekomen kan worden.

Hieruit volgt dat ook Hypothese 2 correct is.

\textbf{Hypothese 3: Het uitvoeren van GeoSPARQL queries op een ``SPARQL endpoint'' is niet vanzelfsprekend. Het is echter mogelijk door de filtering op de client-side uit te voeren.}

Een ``SPARQL endpoint'' is zelfstandig in staat om te antwoorden op een SPARQL query. Hierbij is er geen optie om een GeoSPARQL query door te sturen omdat het ``SPARQL endpoint'' hier niet kan op antwoorden. Bij een ``SPARQL endpoint'' wordt dit probleem aangepakt door niet de volledige query door te sturen, maar in de plaats te tellen hoeveel antwoorden er zijn op elk \textit{triple pattern fragment} in de query. Zo kan de client zelf beslissen welk fragment nodig is om het kleinst mogelijke patroon te vinden. Hierdoor kan het joinen van het resultaat op de client gebeuren. Ook hierbij is dus de laatste stap om het resultaat te filteren op de client. 

Hieruit volgt dat ook Hypothese 3 correct is.

\section{Toekomstig werk}

De gemaakte implementatie is slechts een beperkte implementatie van GeoSPARQL (zoals vermeld in \subsectionref{subsec:toekomstwerk}). Bij verdere implementatie hiervan moet gecontroleerd worden in hoeverre de libraries (zoals ``Turf'' en ``Proj4'') de vereiste functionaliteiten correct ondersteunen. Enerzijds voorziet Turf een groot arsenaal aan geospatiale functionaliteiten, maar wanneer deze niet helemaal kloppen met de verwachtingen is het niet mogelijk om deze manueel aan te passen, om andere conclusies te trekken. Hierbij zou het een meerwaarde zijn om een \textit{library} te maken die eerder werkt op basis van de DE-9IM intersectie. Hiermee wordt bedoeld dat de DE-9IM intersectie aangegeven zou worden door deze nieuwe \textit{library}. Aan de hand van DE-9IM worden alle vereisten van de GeoSPARQL-functionaliteiten uitgedrukt. Zo zou het eenvoudiger zijn om een specifieke functionaliteit van GeoSPARQL correct te implementeren. 

Verder zou het ideaal zijn, moest Sparqlee uitgebreid worden met \textit{custom} functies, zodat de GeoSPARQL-functionaliteiten in Comunica zelf geïmplementeerd kunnen worden. Deze zouden dan geïnjecteerd moeten worden in Sparqlee. Op deze manier kan de modulariteit van Comunica volledig benut worden.

Als laatste blijft het steeds een zoektocht naar de meest performante manier om alle data te verwerken. In deze masterproef werd rekening gehouden met performantie, maar aangezien dit niet de focus was, is hier niet te diep op ingegaan. Zo dient er een benchmarking te gebeuren ter controle of de gemaakte oplossingen schaalbaar zijn.

\section{Tot slot}
In deze masterproef is een implementatie gemaakt van GeoSPARQL. Zoals besproken in \chapterref{chap:implementatie} is deze implementatie grotendeels gemaakt in Sparqlee, met een eigen GeoSPARQL actor in Comunica. Hierbij is gebruik gemaakt van de libraries ``Turf'', ``Proj4'' en ``Terraformer''. Bij deze implementatie is bovendien een client gemaakt, zodat dit geheel in een visuele omgeving zichtbaar is. Deze omgeving voorziet voldoende logs om te controleren hoe alles samenwerkt. Daarnaast is het mogelijk om de performantie te controleren, aangezien deze client meegeeft hoelang de query duurde om uit te voeren. 

Vervolgens werd deze implementatie toegepast in \chapterref{chap:interfaces}. Hierbij werd getest of de verschillende interfaces uitgebreid konden worden met GeoSPARQL-functionaliteiten. In dit hoofdstuk wordt beschreven hoe het geheel in zijn werk gaat. Aan de hand van deze beschrijving wordt nogmaals duidelijk waarom het filteren op de client belangrijk is. 

In \chapterref{chap:conclusie} worden deze resultaten opnieuw geïnterpreteerd. Zo kan een concreet antwoord gevormd worden op de hypotheses die gesteld zijn in \sectionref{sec:onderzoeksvraag}. Tot slot kan geconcludeerd worden dat deze hypotheses correct zijn. Dit betekent dat zowel ``data dumps'', als ``\acrshort{tpf} interfaces'', als ``SPARQL endpoints'' uitgebreid kunnen worden met GeoSPARQL-functionaliteiten door de filtering op de client uit te voeren.
\include{special_chapters/Bibliografie}

\todo{citaat bij begin van een hoofdstuk is fancy}
\todo{zoeken naar data en informatie, mogelijks vervangen door gegevens}
\todo{acroniemen en moeilijke woorden tonen! (TPF, W3C)}
\todo{lijst van woorden aanvullen en naar verwijzen in tekst!}
\todo{captions naar nederlands}
\todo{referenties controleren -> geen vraagtekens!!}
\todo{github verwijzingen naar mijn code toevoegen}
\todo{check of layout abstract NL en EN gelijk is}
\todo{check overfull hboxes}

\end{document}