% !TeX spellcheck = nl_NL
\begin{savequote}[0.55\linewidth]
	``The most efficient network request is one that doesn't use the network.''
	\qauthor{\textasciitilde Roy T. Fielding}
\end{savequote}

\chapter{Resultaten}
\label{chap:resultaten}

Hierna worden telkens de ervaringen van gebruikers besproken. Dit omvat absolute ervaringen (zonder exact referentiepunt, maar tegenover de expected service zoals vermeld in hoofdstuk~\ref{chap:onderzoek}), maar ook de relatieve ervaringen tussen LC2Irail en Linked Connections, en de relatieve ervaringen tegenover de huidige applicatie van de gebruiker zullen besproken worden.

Tot slot zal gekeken worden naar de uiteindelijke keuze van de gebruiker, en worden ook de resultaten van de enquête besproken. Een globale interpretatie van de resultaten volgt in hoofdstuk~\ref{chap:interpretatie}.

Alle ruwe data zijn beschikbaar in de vorm van een spreadsheet: \url{https://github.com/Bertware/masterthesis/blob/master/data.ods?raw=true}.

\section{Liveboards}
\subsection{Metingen}
\begin{figure}[h]
	\centering
	\includegraphics[width=0.80\textwidth]{Optimalisaties_liveboards.eps}
		\caption[Gemeten laadtijd liveboards bij verschillende implementatiedetails]{De gemiddelde gemeten laadtijd voor liveboards gebruikmakend van een HTC 10 voor 262 opzoekingen gebaseerd op de iRail-logs. }
	\label{fig:liveboardlabtest}
\end{figure}
%\begin{table}[h]
%	\begin{tabular}{| c | c | c | c | c | c |}
%		\hline
%		Variant & parser & cache & minimaal (ms) & gemiddelde (ms) & maximaal (ms)\\
%		\hline
%		LC op toestel & org.json & nee & 302 &  595 &  3471 \\
%		LC op toestel & org.json & ja & 409 &  516  &  3599 \\
%		LC op toestel & LoganSquare & nee & 392 & 597 & 2027 \\
%		LC op toestel & LoganSquare & ja  & 166 & 313 & 3428 \\
%		
%		LC op server &&&  232 & 397 &  1421\\
%		\hline
%	\end{tabular}
%	\caption[Gemeten laadtijd liveboards]{De gemeten laadtijd voor het eerste resultaat liveboards gebruikmakend van een HTC 10 voor 262 opzoekingen gebaseerd op de iRail-logs. }
%	\label{tab:liveboardlabtest}
%\end{table}

Zoals eerder vermeld worden eerst kort verschillende implementaties van Linked Connections vergeleken. In grafiek~\ref{fig:liveboardlabtest} zijn de gemiddelde resultaten zichtbaar van een benchmark waarbij 262 stations opgezocht werden, ongeveer 5\% van de opzoekingen door gebruikers op 2 mei 2018. Telkens is de minimale, gemiddelde en maximale responstijd gemeten, dit zowel gebruikmakend van de standaard (\foreign{org.json}) JSON parser en gebruikmakend van de \foreign{LoganSquare}-parser. Ook werd de test herhaald met cache in- en uitgeschakeld, om zo het effect hiervan te meten. Tot slot werd dezelfde test herhaald gebruikmakend van data afkomstig van de LC2Irail web applicatie om een vergelijking tussen de twee methodes te kunnen maken. Deze cijfers geven slechts een indicatie van de snelheid -- een volledige en diepgaande statistische analyse van de performantieverschillen tussen verschillende implementaties van dezelfde techniek valt wegens tijdsgebrek buiten het bereik van deze masterproef.

In deze cijfers is invloed van de cache duidelijk merkbaar. Ook is een duidelijk verschil zichtbaar tussen de JSON parsers: terwijl bij gebruik van de \foreign{LoganSquare}-parser de gemiddelde laadtijd bijna halveert, is het effect van de cache bij het gebruik van de \foreign{org.json} parser veel kleiner. Wanneer de cache uitgeschakeld is, is het verschil tussen de parsers verwaarloosbaar. Dit is mogelijk te verklaren doordat voor het tonen van vertrekken of aankomsten er relatief weinig data nodig is: in de meeste gevallen volstaat een enkele Linked Connections-pagina.

Om een exact beeld te vormen van de prestaties, worden een duizendtal liveboards opgezocht. Hiervoor wordt elke vijfde opzoeking uit de iRail-logs gekozen. Voor elk opzoeking worden 20 resultaten geladen. De resultaten hiervan zijn zichtbaar in grafieken~\ref{fig:liveboardsDiefBest},~\ref{fig:liveboardsDiefAvg} en~\ref{fig:liveboardsDiefSlechtst}, respectievelijk voor het tiende, vijftigste en negentigste percentiel. In deze grafieken worden duidelijke trends zichtbaar:

\begin{figure}[h]
	\centering
	\includegraphics[width=1.00\textwidth]{dief_liveboards_best.eps}
	\caption[Aantal resultaten liveboards in functie van de tijd (10e percentiel)]{Het aantal resultaten in functie van de verlopen tijd.}
	\label{fig:liveboardsDiefBest}
\end{figure}

\begin{figure}[h]
	\centering
	\includegraphics[width=1.00\textwidth]{dief_liveboards_gemiddeld.eps}
	\caption[Aantal resultaten liveboards in functie van de tijd (mediaan)]{Het aantal resultaten in functie van de verlopen tijd.}
	\label{fig:liveboardsDiefAvg}
\end{figure}

\begin{figure}[h]
	\centering
	\includegraphics[width=1.00\textwidth]{dief_liveboards_slechtst.eps}
	\caption[Aantal resultaten liveboards in functie van de tijd (90e percentiel)]{Het aantal resultaten in functie van de verlopen tijd.}
	\label{fig:liveboardsDiefSlechtst}
\end{figure}


\begin{itemize}
    \item In de snelste gevallen is de serverimplementatie sneller. Hiervoor kunnen verschillende oorzaken aangewezen worden: 
	\begin{itemize}
		\item De serverimplementatie kan resultaten op een specifieke vraag cachen, terwijl de lokale implementatie deze steeds zal herberekenen vanaf Linked Connections pagina's. Voor veel voorkomende zoekopdrachten, zoals populaire stations, kan de server ook Linked Connections pagina's cachen.
		\item De serverimplementatie hoeft minder data te versturen. Ook het parsen van het antwoord gaat sneller, gezien slechts een kleine hoeveelheid data verwerkt moet worden en er verder geen berekeningen moeten gebeuren.
	\end{itemize}

	\item Terwijl in de snelste gevallen de serverimplementatie sneller is, is dit verschil beperkt tot ongeveer 60 milliseconden voor het eerste resultaat. Dit ligt zeer kort bij een verschil van 10\%, wat in deze ordegrootte moeilijk onderscheidbaar is voor gebruikers~\citep{miller68}. %TODO CITE Weber-Fechner Law
	
	\item Wanneer naar de mediane performantie gekeken wordt, is Linked Connections duidelijk sneller voor de eerste resultaten. Dit snelheidsverschil is aanzienlijk, en hoogstwaarschijnlijk vooral te wijten aan het feit dat Linked Connections volledige ondersteuning biedt voor incrementele resultaten, waarbij de server meestal meerdere pagina's zal overlopen voor een  antwoord gegeven wordt. Dit is te zien aan de steile curves voor LC2Irail, waar de curves voor LC gekenmerkt worden door een minder sterke stijging. 
	
	Anderzijds is er bij Linked Connections sprake van een \foreign{overhead} door het laden van te veel data. Voor stations met weinig vertrekken, of tijdens piekuren, kan het hierdoor meer moeite kosten om Linked Connections te verwerken. Vermoeden wordt dat dit contrast met de korte, informatiedichte antwoorden van de RPC API zorgt voor het trager laden van resultaten. Ook is zichtbaar dat hoe sneller het toestel is, hoe langer Linked Connections het snelst blijft. Dit bevestigt de hypothese dat het verwerken van Linked Connections pagina's aan de oorzaak ligt. Ook blijft het verschil tussen Linked Connections en LC2Irail beperkt op de HTC 10, terwijl dit verschil aanzienlijk oploopt op de tragere HTC One.
	
	\item In alle gevallen is er een sterke gelijkenis tussen de curves voor LC2Irail op het HTC 10 toestel en het HTC One toestel. Deze zijn enkel een relatief kleine afstand in tijd verschoven, wat verklaard kan worden door de lage belasting voor het mobiele toestel wanneer een RPC API gebruikt wordt.

	\item Het verschil in laadtijd tussen Linked Connections op de twee toestellen loopt lineair op met de benodigde laadtijd om resultaten te laden. Zo is te zien dat in het slechtste geval dubbel zoveel tijd benodigd is op de HTC One in vergelijking met de HTC 10. Dit is een sterke indicatie dat de performantie van Linked Connections afhankelijk is van het gebruikte toestel.
\end{itemize}

\begin{figure}[h]
	\centering
	\includegraphics[trim=3cm 4cm 0 0, width=1.00\textwidth]{boxplot_liveboards_1.eps}
	\caption[Laadtijd eerste resultaat liveboard in functie van toestel en technologie]{Laadtijd eerste resultaat liveboard in functie van toestel en technologie.}
	\label{fig:liveboardsBoxplot1}
\end{figure}

\begin{figure}[h]
	\centering
	\includegraphics[trim=3cm 4cm 0 0, width=1.00\textwidth]{boxplot_liveboards_10.eps}
	\caption[Laadtijd tiende resultaat liveboard in functie van toestel en technologie]{Laadtijd tiende resultaat liveboard in functie van toestel en technologie.}
	\label{fig:liveboardsBoxplot10}
\end{figure}

Wanneer naar de spreiding van de laadtijden gekeken wordt, zichtbaar in figuur~\ref{fig:liveboardsBoxplot1} voor het eerste resultaat en figuur~\ref{fig:liveboardsBoxplot10} voor het tiende resultaat, zijn ook hier duidelijke verschillen tussen de verschillende methodes zichtbaar. Telkens levert LC2Irail een kleinere spreiding verdeling op dan LC, waarbij Linked Connections enkel sneller is voor het eerste resultaat op de HTC 10. Voor het eerste resultaat op de HTC One, en het tiende resultaat op de HTC 10, zijn beide implementaties volgens de wet van Weber-Fechner niet te onderscheiden op de mediaan.

In deze grafieken zijn ook de verschillen tussen toestellen enorm interessant. Zo is te zien dat bij gebruik van een RPC API de mediaan van de laadtijd ongeveer gelijk is tussen verschillende toestellen, en de interkwartielafstand relatief klein is, wat op een kleine spreiding en dus consistente resultaten per toestel duidt. De mediaan voor beide toestellen ligt respectievelijk op 578 en 650 milliseconden voor het eerste resultaat. Volgens de wet van Weber-Fechner is dit onmerkbaar voor gebruikers. Ook het eerste kwartiel ligt kort genoeg bij elkaar om niet merkbaar te zijn. Ondanks dat boven de mediaan opzoekingen via LC2Irail op de HTC One meer tijd vergen, is onder de mediaan voor gebruikers geen verschil merkbaar. Deze zeer consistente ervaring over toestellen is zeer wenselijk voor routeplanning applicatie. Voor het tiende resultaat neemt het verschil tussen de toestellen toe, waarbij gebruikers een verschil zullen merken tussen de LC2Irail implementatie op verschillende toestellen.

Bij gebruik van Linked Connections zijn duidelijke verschillen tussen de mediaan van de laadtijd te zien bij verschillende toestellen. Ook de interkwartielafstand varieert tussen toestellen: zo is een ouder toestel niet enkel trager, maar is ook de spreiding veel groter, en zijn de resultaten dus minder consistent op oudere (tragere) toestellen. Anderzijds is Linked Connections in de meeste gevallen wel duidelijk sneller dan LC2Irail op de HTC 10.

Een eigenaardigheid is dat de minima voor de HTC One telkens lager liggen. Dit wordt vermoedelijk veroorzaakt door verschillende Android versies, met een verschillende aanpak op vlak van asynchrone scheduling.

\subsection{Ervaringen}
Wanneer naar de ervaringen van gebruikers gekeken wordt, stemmen deze ongeveer overeen met wat verwacht zou worden na evaluatie van de metingen.

Uit figuren~\ref{fig:liveboardsBoxplot1} en~\ref{fig:liveboardsBoxplot10} blijkt dat de performantie van LC2Irail consistenter is, zowel in de vorm van een kleinere spreiding van de resultaten op eenzelfde toestel, als in de vorm van kleinere verschillen tussen toestellen. Ook bij de gebruikerservaring komt deze kleine spreiding terug. Wanneer de ervaren snelheid wordt uitgezet in een boxplot per techniek, zichtbaar in figuur~\ref{fig:liveboardsUx}, blijkt dat net als bij de testen, LC2Irail een heel consistente beoordeling wordt gegeven. Voor Linked Connections zijn de beoordelingen meer gespreid en liggen deze iets lager.

\begin{figure}[h]
	\centering
	\includegraphics[trim=3cm 4cm 0 0, width=0.80\textwidth]{boxplot_liveboards_ux.eps}
	\caption[Ervaren snelheid van liveboards]{De ervaren snelheid op een schaal 1-7 van vertrekken en aankomsten voor LC2Irail en Linked Connections, gebaseerd op 17 user-tests.}
	\label{fig:liveboardsUx}
\end{figure}

Hoewel 12 van de 17 testpersonen Linked Connections als redelijk tot extreem snel ervaart, zijn er slechts twee personen die Linked Connections sneller ervaren dan LC2Irail. De ervaringen en uiteindelijk keuze op basis van snelheid is zichtbaar in figuur~\ref{fig:alluvialUserChoicesLiveboards}. Het is duidelijk zichtbaar dat de ervaringen voor Linked Connections veel gemengder zijn dan de ervaringen voor LC2Irail. Er zijn zowel mensen die Linked Connections sneller, even snel of trager ervaren. 

\begin{figure}[ht]
	\centering
	\includegraphics[width=0.90\textwidth]{alluvial_user_choice_departures.eps}
	\caption[Door gebruikers gekozen implementatie voor liveboards]{Verbanden tussen de door gebruikers gekozen implementaties voor liveboards. }
	\label{fig:alluvialUserChoicesLiveboards}
\end{figure}

Alle testpersonen werden expliciet gevraagd welke implementatie ze als sneller ervoeren. Hierbij waren de antwoorden verdeeld: acht personen kozen de Linked Connections, vier personen hadden geen mening, en vijf personen kozen LC2Irail variant. Het is moeilijk om hier onmiddellijk conclusies uit te trekken. Om deze reden zullen de resultaten in hoofdstuk~\ref{chap:interpretatie}algemener besproken worden. Wel kan gezegd worden dat de gebruikerservaringen zeer gemengd zijn voor Linked Connections, en zal Linked Connections sowieso niet voor de volledige populatie sneller zijn.

Wanneer gekeken wordt naar de verschillen tussen de JSON parsers, blijkt dat beide parsers ongeveer even goed presteren in de ogen van de testers. Respectievelijk 5 op 7 en 7 op 10 testers zijn neutraal of tevreden, en bij beide varianten is er telkens een tester neutraal. Deze vergelijking is echter slechts een indicatie, en is door te kleine steekproeven ongeschikt om te veralgemenen naar een grotere populatie.

Wanneer er echter gekeken wordt naar de gemeten prestatieverschillen tussen beide JSON parsers tijdens de user-tests, is er een duidelijk verschil, waarbij het 90e percentiel van de laadtijd onder de LoganSquare parser lager ligt dan de mediane laadtijd van de org.json parser. Dit verschil lijkt echter geen invloed te hebben op de ervaringen van gebruikers, vermoedelijk omdat men beide reeds als performant genoeg ervaart. % TODO: Hoe groot procentueel verschil?

\begin{figure}[ht]
	\centering
	\includegraphics[width=1.00\textwidth]{userdata_liveboards_currentapp.eps}
	\caption[Door gebruikers ervaren snelheid liveboards tov huidige apps]{De door 17 gebruikers ervaren snelheid liveboards ten opzichte huidige apps }
	\label{fig:relativePerceptionLiveboards}
\end{figure}

Als tot slot gevraagd wordt om de snelheid te vergelijken met de applicatie die de gebruiker op dit moment gebruikt, zichtbaar in figuur~\ref{fig:relativePerceptionLiveboards} , komen beide apps er goed uit. LC2Irail scoort hier zeer goed, en ondanks dat Linked Connections iets minder goed scoort dan LC2Irail, geven slechts 2 gebruikers aan Linked Connections trager te ervaren dan hun huidige applicatie.

\section{Routes}

\subsection{Metingen}
\begin{figure}[h]
	\centering
	\includegraphics[width=0.80\textwidth]{Optimalisaties_routes.eps}
	\caption[Gemeten laadtijd routes bij verschillende implementatiedetails]{De gemiddelde gemeten laadtijd voor routes gebruikmakend van een HTC 10 voor 779 opzoekingen gebaseerd op de iRail-logs.}
	\label{fig:routelabtest}
\end{figure}
%\begin{table}[h]
%	\begin{tabular}{| c | c | c | c | c | c |}
%		\hline
%		Variant & parser & cache & minimaal (ms) & gemiddelde (ms) & maximaal (ms)\\
%		\hline
%		LC op toestel & org.json & nee & 401 & 3539 & 8531\\
%		LC op toestel & org.json & ja & 221 & 2172 & 6960 \\
%		LC op toestel & LoganSquare & nee &  386 & 3374 & 7554 \\
%		LC op toestel & LoganSquare & ja  & 233 & 1973 & 7640 \\
%		
%		LC op server &&&  27 & 1126 & 3374\\
%		\hline
%	\end{tabular}
%	\caption[Gemeten laadtijd routes]{De gemeten laadtijd voor routes gebruikmakend van een HTC 10 voor 779 opzoekingen gebaseerd op de iRail-logs.}
%	\label{tab:routelabtest}
%\end{table}

Ook voor routes zullen eerst de relatieve prestaties van verschillende implementaties kort besproken worden. In grafiek~\ref{fig:routelabtest} zijn de gemiddelde resultaten zichtbaar van een benchmark waarbij 779 routes opgezocht werden, ongeveer 5\% van de opzoekingen door gebruikers op 2 mei 2018. Telkens is de minimale, gemiddelde en maximale responstijd gemeten, dit zowel gebruikmakend van de standaard JSON parser (\foreign{org.json}) en gebruikmakend van de \foreign{LoganSquare}-parser. Ook werd de test herhaald met cache in- en uitgeschakeld, om zo het effect hiervan te meten. Tot slot werd dezelfde test herhaald gebruikmakend van data afkomstig van de LC2Irail web applicatie om een vergelijking tussen de twee methodes te kunnen maken. Deze cijfers geven slechts een indicatie van de snelheid - een volledige en diepgaande statistische analyse van de performantieverschillen tussen verschillende implementaties van dezelfde techniek valt wegens tijdsgebrek buiten het bereik van deze masterproef.

Net zoals bij liveboards is ook hier de invloed van de cache duidelijk merkbaar. Dit valt te verklaren door de grote hoeveelheden data die verwerkt moeten worden, waarbij het cruciaal is dat deze niet steeds opnieuw gedownload wordt. Vergeleken met dezelfde analyse voor Liveboards (figuur~\ref{fig:liveboardlabtest}), is er hier een minder groot verschil tussen de parsers. Dit komt mogelijk door de grotere impact van de verdere algoritmes, waardoor de invloed van het parsen verkleint. Het algoritme om de data tot routes te verwerken is het zwaarst van de drie endpoints. %TODO: remove LC2Irail from graph

Om ook hier een exact beeld te vormen van de prestaties, worden opnieuw een duizendtal zoekopdrachten uitgevoerd. Hiervoor kiezen wordt elke vijfde opzoeking uit de iRail-logs gekozen. Voor elke route wordt gepoogd 10 resultaten te laden. De resultaten hiervan zijn zichtbaar in grafieken~\ref{fig:routesDiefBest},~\ref{fig:routesDiefAvg} en~\ref{fig:routesDiefSlechtst}, respectievelijk voor het tiende, vijftigste en negentigste percentiel.

\begin{figure}[h]
	\centering
	\includegraphics[width=1.00\textwidth]{dief_routes_best.eps}
	\caption[Aantal resultaten routes in functie van de tijd (10e percentiel)]{Het aantal resultaten in functie van de verlopen tijd.}
	\label{fig:routesDiefBest}
\end{figure}

\begin{figure}[h]
	\centering
	\includegraphics[width=1.00\textwidth]{dief_routes_gemiddeld.eps}
	\caption[Aantal resultaten routes in functie van de tijd (mediaan)]{Het aantal resultaten in functie van de verlopen tijd.}
	\label{fig:routesDiefAvg}
\end{figure}

\begin{figure}[h]
	\centering
	\includegraphics[width=1.00\textwidth]{dief_routes_slechtst.eps}
	\caption[Aantal resultaten routes in functie van de tijd (90e percentiel)]{Het aantal resultaten in functie van de verlopen tijd.}
	\label{fig:routesDiefSlechtst}
\end{figure}

Uit deze drie grafieken kunnen opnieuw enkele conclusies getrokken worden:
\begin{itemize}
	\item In alle grafieken en voor alle testen, hebben de curves een gelijkaardige vorm, waarbij er na een relatief lange wachttijd aan snel tempo resultaten geladen worden: in het geval van Linked Connections is voor de eerste opzoeking telkens een relatief grote hoeveelheid data nodig is, waarna slechts één of twee extra pagina's moeten opgehaald worden om het volgend resultaat te bepalen. In het geval van LC2Irail worden resultaten in grote blokken binnengehaald, waarbij vanaf de tweede opzoeking reeds veel data in cache zitten. In het geval van LC2Irail worden resultaten ook onmiddellijk voor grote intervallen opgehaald, om zo het aantal verzoeken te beperken. 
	\item Terwijl op de HTC 10 Linked Connections in alle gevallen beter presteert dan LC2Irail, presteert Linked Connections slechter dan LC2Irail op de HTC One. 
	\item Opnieuw presteert LC2Irail op beide toestellen gelijkaardig, met slechts een kleine verschuiving in tijd tussen beide curves.
	\item Terwijl in het slechtste geval bijna alle varianten gelijk presteren, loopt Linked Connections op de HTC One enorm achter. Uit alle grafieken volgt dat hoe trager het toestel, hoe trager Linked Connections, terwijl LC2Irail ongeveer even goed blijft presteren. Bijgevolg kan dus ook gesteld worden dat alle toestellen trager dan de HTC10 in het slechtste geval trager zullen presteren dan de HTC One.
	\item In alle gevallen laadt het eerste resultaat pas na anderhalve tot zeven seconden. Dit zijn zeer lange laadtijden, waarvan verwacht wordt dat ze de gebruikerservaring negatief gaan beïnvloeden.
\end{itemize}

Wanneer specifiek naar de verdelingen gekeken wordt, gevisualiseerd door middel van de boxplots in figuur~\ref{fig:routesBoxplot1} en~\ref{fig:routesBoxplot10}, zijn sterke verschillen te zien, zowel tussen toestellen als implementaties. Net als bij liveboards ook hier duidelijk te zien hoe LC2Irail gelijke prestaties heeft op beide toestellen, met bijna identieke distributies. Dit in tegenstelling tot de prestaties van Linked Connections, die zeer sterk variëren per toestel. Op de HTC 10 zal voor al meer dan 75\% van de opzoekingen via Linked Connections het eerste resultaat geladen zijn op het moment dat LC2Irail op hetzelfde toestel minder dan 25\% van de verzoeken beantwoord heeft. Op het HTC One toestel is dit echter omgekeerd, en nog extremer. 

Voor het tiende resultaat zijn deze verschillen tussen toestellen iets minder extreem, al zijn ze nog steeds zeer groot. Linked Connections blijft iets consistentere resultaten bieden op de HTC 10, terwijl de HTC One net minder consistente resultaten biedt dan LC2Irail. LC2Irail biedt ook voor het tiende resultaat exact dezelfde ervaring op beide toestellen, terwijl het verschil tussen de mediaan voor Linked Connections op beide toestellen niet acceptabel is en de gebruikerservaring duidelijk zal beïnvloeden.

\begin{figure}[h]
	\centering
	\includegraphics[trim=3cm 4cm 0 0, width=0.80\textwidth]{boxplot_routes_1.eps}
	\caption[Laadtijd eerste resultaat route in functie van toestel en technologie]{Laadtijd eerste resultaat route in functie van toestel en technologie.}
	\label{fig:routesBoxplot1}
\end{figure}

\begin{figure}[h]
	\centering
	\includegraphics[trim=3cm 4cm 0 0, width=0.80\textwidth]{boxplot_routes_10.eps}
	\caption[Laadtijd tiende resultaat route in functie van toestel en technologie]{Laadtijd tiende resultaat route in functie van toestel en technologie.}
	\label{fig:routesBoxplot10}
\end{figure}

\subsection{Ervaringen}

Op vlak van gebruikerservaring wordt verwacht dat gebruikers net zoals bij Liveboards de implementatie op basis van LC2Irail consistenter zullen beoordelen, en dat, afhankelijk van het door de tester gebruikte toestel, Linked Connections sneller, even snel of trager dan LC2Irail ervaren wordt.

\begin{figure}[h]
	\centering
	\includegraphics[trim=3cm 4cm 0 0, width=0.80\textwidth]{boxplot_routes_ux.eps}
	\caption[Ervaren snelheid van routes]{De ervaren snelheid op een schaal 1-7 van routes voor LC2Irail en Linked Connections, gebaseerd op 17 user-tests.}
	\label{fig:routesUx}
\end{figure}

Wanneer de resultaten van user-testing vergeleken worden met de verwachtingen, blijken deze verwachtingen grotendeels in vervulling te gaan. In figuur~\ref{fig:routesUx} is te zien dat de prestaties van LC2Irail iets consistenter goed beoordeeld worden, en LC2Irail een betere beoordeling krijgt dan Linked Connections. In vergelijking met liveboards (figuur~\ref{fig:liveboardsUx}) is te zien dat gebruikers bij routes iets minder verdeeld zijn over de prestaties van Linked Connections, al komt dit omdat er geen gebruikers voor de hoogste score kiezen. 

De consistentere beoordelingen voor Linked Connections zijn ook duidelijk wanneer dieper ingegaan wordt op hoe eenzelfde gebruiker de snelheid van LC2Irail en Linked Connections ervaart, zichtbaar in figuur~\ref{fig:alluvialUserChoicesRoutes}. Hoewel veel gebruikers Linked Connections als trager ervaren, is er ook een klein aantal gebruikers dat Linked Connections sneller ervaart. Het zijn deze personen die, wanneer expliciet gevraagd wordt om de snelste implementatie aan te duiden, voor Linked Connections kiezen. Dit duidelijk verband kon bij liveboards niet gelegd worden (figuur~\ref{fig:alluvialUserChoicesLiveboards}), wat te wijten kan zijn aan een voor gebruikers slechts klein, onduidelijk verschil tussen de prestaties bij liveboards. Voor routes wordt het verschil veel duidelijker ervaren, en kiezen gebruikers dus consistenter met de door hun ervaren snelheden.
Opvallend is ook dat er een persoon is die Linked Connections als zeer snel bestempelt, en Linked Connections hiermee even snel of sneller dan LC2Irail ervaart. Echter kiest deze persoon toch voor LC2Irail wanneer een expliciete keuze gemaakt dient te worden. Er kan gesteld worden dat voor deze gebruiker het verschil niet merkbaar was. Aan de andere kant kiest iedereen die Linked Connections als traag of gemiddeld ervaart voor LC2Irail, op één gebruiker na die onbeslist is. Voor deze gebruikers moet Linked Connections nog sneller gemaakt worden alvorens het concurrentieel wordt.

\begin{figure}[ht]
	\centering
	\includegraphics[width=0.90\textwidth]{alluvial_user_choice_routes.eps}
	\caption[Door gebruikers gekozen implementatie voor routes]{Verbanden tussen de door gebruikers gekozen implementaties voor routes. }
	\label{fig:alluvialUserChoicesRoutes}
\end{figure}

Wanneer voor routes beide JSON parsers vergeleken worden, is te zien dat voor de \foreign{LoganSquare}-parser de proefpersonen een meer uitgesproken mening hadden: er waren zowel meer tevreden als ontevreden personen, terwijl bij de \foreign{org.json} parser veel mensen neutraal waren. Dit gaat echter in tegen een praktijktest waarbij enkele gebruikers achtereenvolgens een versie gebruikmakend van de \foreign{org.json} en \foreign{LoganSquare}-parser voorgeschoteld kregen, gaven deze telkens aan de versie op basis van \foreign{LoganSquare} sneller te ervaren, zowel op goedkope als dure smartphones. Hieruit kan besloten worden dat de user-tests, opgedeeld per parser, te kleine steekproeven zijn om een algemene conclusie te vormen over de invloed van de parsers.

\begin{figure}[ht]
	\centering
	\includegraphics[width=1.00\textwidth]{userdata_routes_currentapp.eps}
	\caption[Door gebruikers ervaren snelheid routes tov huidige apps]{De door 17 gebruikers ervaren snelheid routes ten opzichte huidige apps }
	\label{fig:relativePerceptionRoutes}
\end{figure}

Als tot slot gevraagd wordt om de snelheid te vergelijken met de applicatie die de gebruiker op dit moment gebruikt, doen beide testversies het goed ten opzichte van de huidige applicaties. Dit is zichtbaar in figuur~\ref{fig:relativePerceptionRoutes}. Linked Connections scoort iets minder goed dan LC2Irail, maar de overgrote meerderheid vindt Linked Connections nog steeds even snel of sneller dan de applicatie die men gewoonlijk gebruikt.


\section{Voertuigen}

\subsection{Metingen}
Het opzoeken van het traject dat een voertuig aflegt verschilt sterk van de eerder besproken endpoints. Het traject van een voertuig wordt door HyperRail als één (ondeelbaar) resultaat beschouwd, en kan hierdoor niet per stop geladen worden. Dit in tegenstelling tot liveboards en routes, waar er verschillende resultaten zijn die elk onafhankelijk van elkaar zijn.

Dit is ook de opzoeking die het meeste data vereist bij Linked Connections: alle pagina's moeten doorzocht worden op connecties met betrekking tot één specifiek voertuig. Dit voertuig komt slechts in een relatief beperkt aantal pagina's voor, gezien het voertuig slechts enkele uren rijdt, en het tijdstip van vertrek en aankomst onbekend zijn. Zoals eerder vermeld %TODO: referntie %TODO: aantal stops
zijn hier enkele oplossingen voor, zoals het gebruik van een index. In deze context wordt een index gedefinieerd als een lijst van alle treinen voor een bepaalde periode (in dit geval mei 2018) en het tijdstip van hun eerste vertrek.

Om een idee te krijgen van de invloed van deze index, alsook van het gebruik van een cachegeheugen voor de Linked Connections pagina's bij deze opzoekingen, werden 102 voertuigen opgezocht, voor alle combinaties van cache en index gebruik. Tevens werd een extra test gedaan met een cache die in het RAM-geheugen geplaatst wordt (in tegenstelling tot het flashgeheugen van het toestel), en een vergelijkende test waarbij de Linked Connections server gebruikt werd. De gemiddelde zoektijd hiervoor is gevisualiseerd in figuur~\ref{fig:vehiclelabtest}.

\begin{figure}[h]
	\centering
	\includegraphics[width=1.00\textwidth]{Optimalisaties_voertuigen.eps}
	\caption[Gemeten laadtijd voertuigen bij verschillende implementatiedetails]{De gemeten laadtijd voor voertuigen gebruikmakend van een HTC 10 voor 102 opzoekingen gebaseerd op de iRail-logs. }
	\label{fig:vehiclelabtest}
\end{figure}

%\begin{table}[ht]
%	\begin{tabular}{| c | c | c | c | c | c | c |}
%		\hline
%		Variant & parser & cache & index & minimaal (ms) & gemiddelde (ms) & maximaal (ms)\\
%		\hline
%		LC op toestel & org.json & nee & nee & 540 & 6764 & 12676 \\
%		LC op toestel & org.json & ja & nee & 483 & 6488 & 10921 \\
%		
%		LC op toestel & org.json & nee & ja & 2638 & 4443 & 10956 \\
%		LC op toestel & org.json & ja & ja &  2440 & 4066 & 6003\\
%		LC op toestel & org.json & RAM & ja & 2263 & 3912 & 5763 \\
%		LC op toestel & LoganSquare & nee & ja &  1860 & 3283 & 5374 \\
%		LC op toestel & LoganSquare & ja  & ja & 1195 & 1925 & 2888 \\
%		
%		LC op server &&&&  264 & 713 & 5068 \\
%		\hline
%	\end{tabular}
%	\caption[Gemeten laadtijd voertuigen]{De gemeten laadtijd voor voertuigen gebruikmakend van een HTC 10 voor 102 opzoekingen gebaseerd op de iRail-logs. }
%	\label{tab:vehiclelabtest}
%\end{table}

Het is duidelijk dat de standaard implementatie zeer slecht presteert. Ook het gebruik van een cachegeheugen brengt hierbij niet veel beterschap. Wanneer echter een index toegevoegd wordt, is een drastische verbetering merkbaar. Het gemiddelde daalt in deze beperkte test met ongeveer een derde. Toevoeging van een cachegeheugen, op flash of in het RAM-geheugen, brengt ook hier relatief weinig beterschap. 

Een tweede grote verbetering kan behaald worden door het gebruik van de eerder besproken \foreign{LoganSquare}-parser. Hierbij is ook een veel grotere verbetering te zien door cachegebruik dan bij de \foreign{org.json} parser. Dit is logisch, gezien bij het gebruik van de \foreign{LoganSquare}-parser het verwerken van de data relatief gezien minder tijd in beslag neemt - het ophalen van data wordt dus belangrijker. Op het eerste zicht blijven alle lokale varianten veel trager dan de serverimplementatie, die sneller door pagina's kan zoeken.

Het verschil tussen de lokale implementatie en de serverimplementatie wordt nu in detail onderzocht. Hiervoor worden 2102 ritten opgezocht die plaatsvinden op 6 mei 2018. Dit wordt enerzijds gedaan voor de Linked Connections implementatie die gebruikmaakt van de \foreign{LoganSquare}-parser, cache en lokale index, en anderzijds voor de serverimplementatie, die server-side over dezelfde index en een cache beschikt.

Wanneer gekeken wordt naar de boxplot van de responstijd, weergegeven in figuur~\ref{fig:vehicleboxplot}, is te zien dat de Linked Connections duidelijk slechter presteert. Op beide toestellen is LC2Irail sneller. Bij de HTC 10, valt dit nog enigszins mee, maar op de HTC One zijn via LC2Irail de meeste resultaten binnen 3000 milliseconden geladen, terwijl op dat moment nog geen 25\% van de opzoekingen via Linked Connections uitgevoerd werd. Net zoals bij liveboards en routes is hier te zien dat LC2Irail consistente prestaties biedt: beide boxplots zijn ongeveer identiek, op wat uitlopers na. Voor Linked Connections is echter te zien dat, net zoals voor het opzoeken van liveboards en routes, de spreiding van de benodigde tijd afhangt van het toestel: een traag toestel zal niet alleen trager resultaten laden, maar heeft ook een grotere variatie in de laadtijd.

\begin{figure}[h]
	\centering
	\includegraphics[trim=3cm 4cm 0 0, width=1.00\textwidth]{boxplot_vehicles.eps}
	\caption[Prestaties voor het laden van voertuigen]{De prestaties voor het laden van voertuigen, gemeten door alle voertuigen, beschreven in Linked Connections, voor 6 mei op te zoeken.}
	\label{fig:vehicleboxplot}
\end{figure}

%\begin{figure}[h]
%	\centering
%	\includegraphics[width=1.00\textwidth]{distribution_vehicle_loading_cummulatief.eps}
%	\caption[Cummulatieve kans op laden van voertuig]{De kans dat een voertuig geladen is in functie van de verlopen tijd.}
%	\label{fig:vehiclecummulatief}
%\end{figure}

\subsection{Ervaringen}
Wanneer de resultaten van de user-testing bekeken worden, is zoals verwacht te zien dat het laden van voertuigen beduidend slechter scoort wanneer de lokale Linked Connections implementatie gebruikt wordt, vergeleken met wanneer de serverimplementatie gebruikt wordt. In figuur~\ref{fig:vehicleboxplot} is dit duidelijk zichtbaar. Zo beoordelen de meeste gebruikers Linked connections slechts als "gemiddeld", terwijl de meerderheid van de gebruikers de LC2Irail variant als 'Zeer snel` bestempelde. Ook is hier te zien, net als bij liveboards en routes, dat er voor Linked Connections een veel grotere spreiding is in de gegeven antwoorden, terwijl bij LC2Irail iedereen het er over eens lijkt dat deze implementatie snel is.

\begin{figure}[h]
	\centering
	\includegraphics[trim=3cm 4cm 0 0, width=1.00\textwidth]{boxplot_vehicles_ux.eps}
	\caption[Ervaren snelheid van routes]{De ervaren snelheid op een schaal 1-7 van routes voor LC2Irail en Linked Connections, gebaseerd op 17 user-tests.}
	\label{fig:vehiclesUx}
\end{figure}

Wanneer de testgroep van Linked Connections opgesplitst wordt per gebruikte parser, blijkt dat personen die de lokale implementatie op basis van \foreign{LoganSquare} testten, de laadtijd iets beter beoordeelden vergeleken met de groep die de implementatie op basis van de \foreign{org.json} parser testte. Deze resultaten liggen ook in lijn met ervaringen van testers die beide parsers achtereenvolgens voorgeschoteld kregen, waarbij alle testers de \foreign{LoganSqare} parser sneller ervoeren. Ondanks dat de testgroep onvoldoende groot was om een veralgemening te kunnen maken, kan in combinatie met de directe vergelijking wel gesteld worden dat er een grote kans is dat verbeteringen in de implementatie de snelheid verder omlaag kunnen brengen, en zo de gebruikerservaring kunnen verbeteren. Gezien bij het berekenen van voertuigen het meeste data nodig is, is hier de impact van implementatiedetails het grootst.

\begin{figure}[ht]
	\centering
	\includegraphics[width=0.90\textwidth]{alluvial_user_choice_vehicles.eps}
	\caption[Door gebruikers gekozen implementatie voor voertuigen]{Verbanden tussen de door gebruikers gekozen implementaties voor voertuigen. }
	\label{fig:alluvialUserChoicesVehicles}
\end{figure}

Wanneer de gebruiker gevraagd werd te kiezen, koos slechts één gebruiker voor de lokale implementatie in dit onderdeel. Vijf gebruikers hadden geen specifieke voorkeur voor een specifieke implementatie, ook al beoordeelden vier van hen Linked Connections als trager. In figuur~\ref{fig:alluvialUserChoicesVehicles} zijn de ervaringen van elke gebruiker duidelijk te zien. Zo is te zien dat de ervaring voor gebruikers nooit verbeterd, en veel gebruikers een groot verschil ervaren tussen de snelheid van beide implementaties, in het nadeel van Linked Connections. Veel gebruikers die Linked Connections als redelijk of zeer snel ervaren, ervoeren LC2Irail nog steeds als sneller, waardoor ze wanneer ze moesten kiezen niet voor Linked Connections kozen. De enigste gebruiker die voor Linked Connections koos, ervoer Linked Connections niet als sneller dan LC2Irail, maar vond beide wel snel.

\begin{figure}[ht]
	\centering
	\includegraphics[width=1.00\textwidth]{userdata_vehicles_currentapp.eps}
	\caption[Door gebruikers ervaren snelheid voertuigen tov huidige apps]{De door 17 gebruikers ervaren snelheid routes ten opzichte huidige apps }
	\label{fig:relativePerceptionVehicles}
\end{figure}

Als gevraagd wordt om de snelheid te vergelijken met de applicatie die de gebruiker op dit moment gebruikt, geven zes op tien gebruikers aan Linked Connections als even snel te ervaren als hun huidige applicatie voor het opzoeken van voertuigen. De andere gebruikers geven aan de opzoekingen redelijk wat trager tot redelijk wat sneller te ervaren. Dit is duidelijk zichtbaar in figuur~\ref{fig:relativePerceptionVehicles}. Terwijl LC2Irail acceptabel scoort, blijkt Linked Connections hier toch achterop te raken, zowel ten opzichte van LC2Irail als ten opzichte van de Linked Connections prestaties voor voertuigen (figuur~\ref{fig:relativePerceptionLiveboards}) en routes (figuur~\ref{fig:relativePerceptionRoutes}).

\section{Door de gebruiker gekozen implementatie}

Voor alle soorten informatie (liveboards, routes, en voertuigen) lijkt Linked Connections een gelijkaardige of slechtere gebruikerservaring op te leveren dan LC2Irail in termen van laadtijd. Hierbij dient opgemerkt te worden dat dit verschil bij liveboards slechts zeer beperkt is, en het laden nog steeds als snel werd ervaren. Voor routes bestempelden enkele personen Linked Connections als traag, maar ook hier blijft het verschil beperkt. Bij voertuigen blijkt echter dat Linked Connections door drie kwart van de gebruikers als trager werd ervaren, waarbij Linked Connections niet enkel relatief slechter scoort, maar ook in absolute termen slechts door een minderheid van de gebruikers als snel wordt ervaren. 

Linked Connections verschilt echter niet enkel in termen van laadtijd. Zoals eerder aangehaald in hypothese 1 is offline opzoeken mogelijk, en wordt vermoed dat dit de keuze van de gebruiker beïnvloedt. Om dit na te gaan werden de keuzes van gebruikers gevisualiseerd in figuur~\ref{fig:alluvialUserChoices}. In dit diagram is zowel te zien hoeveel gebruikers voor elke implementatie kozen, maar ook hoe de keuze van gebruikers evolueert. Zo is te zien dat naarmate de relatieve prestaties van LC ten opzichte van LC2Irail dalen, personen die eerder voor Linked Connections kozen overstappen op LC2Irail.

\begin{figure}[ht]
	\centering
	\includegraphics[width=0.90\textwidth]{alluvial_user_choice.eps}
	\caption[Door gebruikers gekozen implementatie]{Verbanden tussen de door gebruikers gekozen implementaties voor alle soorten informatie, alsook de resulterende keuze waarbij ook offline toegang in rekening werd gebracht. }
	\label{fig:alluvialUserChoices}
\end{figure}

Terwijl de meerderheid van de gebruikers steeds voor LC2Irail koos, kantelt deze balans volledig om wanneer gebruikers worden gevraagd om met alle aspecten rekening te houden. Dit is duidelijk zichtbaar aan de rechterkant van figuur~\ref{fig:alluvialUserChoices}. Hieruit blijkt dat gebruikers enige snelheid willen opgeven in ruil voor offline opzoekingen. Zeven gebruikers laten weten dat een hybride systeem ideaal zou zijn, waarbij de snelheid van LC2Irail gecombineerd wordt met Linked Connections als offline alternatief. Wanneer deze gebruikers alsnog verplicht werden te kiezen, waren hun keuzes gelijk verdeeld, afhankelijk van de persoonlijke nood om offline te kunnen opzoeken. 

Er wordt nu getracht een antwoord te vinden op de in hoofdstuk~\ref{chap:onderzoek} gestelde vragen omtrent de gebruikerservaring.
\begin{itemize}
	\item Ervaart de gebruiker een app die lokaal Linked Connections gebruikt als sneller dan een app die gebruikmaakt van een RPC API?\\
	De ervaring van de gebruiker hangt sterk af van het gebruikte toestel en de gemaakte opzoekingen. Enkel gebruikers van snelle toestellen ervaren Linked Connections in sommige gevallen als sneller. Voor liveboards is Linked Connections concurrentieel, maar voor andere opzoekingen is deze techniek meestal merkbaar trager dan de gebruikte RPC API.
	\item Ervaart de gebruiker een app die lokaal Linked Connections gebruikt als sneller dan zijn huidige app?\\
	Een minderheid ervaart Linked Connections als sneller dan de huidige gebruikte applicatie. Voor liveboards, routes en voertuigen vinden respectievelijk 45\%, 30\% en 24\% dat Linked Connections sneller is. Respectievelijk 30\%, 40\% en 60\% van de gebruikers ervaren Linked Connections als even snel, met 24\%, 30\% en 12\% van de gebruikers die Linked Connections expliciet als trager ervaren dan hun huidige applicatie. Linked Connections brengt op dit moment zeker geen verbetering in snelheid voor de gebruikers.
\end{itemize}

Wat dit wil zeggen voor de onderzoeksvraag en hypotheses en zal verder besproken worden in hoofdstuk~\ref{chap:interpretatie}.

\section{Enquête}
Zoals vermeld in hoofdstuk~\ref{chap:onderzoek}, wordt getracht een aantal vragen te beantwoorden aan de hand van een enquête. In bijlage~\ref{appendix:enquete} werden alle vragen opgelijst, in bijlage~\ref{appendix:report} werden alle antwoorden toegevoegd. Eerst zullen de resultaten van deze enquête overlopen worden, alvorens aan de hand van deze resultaten een antwoord te zoeken op de in hoofdstuk~\ref{chap:onderzoek} geformuleerde vragen.

\subsection{Antwoorden van respondenten}
Er werden in totaal \emph{81 volledig ingevulde enquêtes} digitaal verzameld, in een gevarieerd publiek. Zo neemt meer dan de helft van de respondenten meerdere keren per week (26\%) of dagelijks (28\%) de trein. Ook personen die occasioneel reizen zijn goed vertegenwoordigd, zo reist 16\% minder dan één keer per maand per trein.

Wanneer gekeken wordt naar de \emph{informatiebronnen} voor reizigers, blijkt dat \foreign{native} applicaties voorop staan (93\%, waarvan 27\% third-party applicaties zijn), gevolgd door digitale informatieborden en affiches (74\%) en websites (62\%) . Hierbij moeten opgemerkt worden dat de enquête specifiek gericht is op personen die applicaties gebruiken om informatie op te zoeken, en het werkelijk aandeel van de reizigers die applicaties gebruikt dus iets lager kan liggen.

Voor iets minder dan de helft (48\%) van de bevraagde reizigers verloopt elke reis probleemloos. Alle respondenten geven aan vertragingen te ervaren bij het reizen, waarbij 33\% aangeeft dat dit zelfs meestal of altijd het geval is. Na vertragingen zijn spoorwijzigingen de tweede grootste bron van ergernis: meer dan 85\% van de gebruikers ervaart dit wel eens, waarbij dit voor 15\% van de reizigers regelmatig voorvalt. Tot slot geeft 72\% van de respondenten aan wel eens een afgeschafte trein te hebben, al gebeurt dit voor slechts 1\% de helft van de tijd.

Wanneer nagegaan wordt hoe \emph{up-to-date informatie} voor reizigers is, wordt duidelijk dat hier zeker ruimte is voor verbetering: slechts 25\% zegt altijd over actuele informatie in de stations te beschikken, applicaties doen het iets beter, waarbij 40\% van de gebruikers altijd over actuele informatie beschikt. Voor beide blijkt dat 50\% van de personen die dit ervaart, er slechts soms last van heeft. Echter blijkt wel dat 27\% van de personen minstens de helt van de tijd dit probleem in stations ervaart. Applicaties doen het iets beter, waar slechts 10\% van de gebruikers dit probleem minstens de helft van de tijd ervaart. Uit deze cijfers kan geconcludeerd worden dat gebruikers wel degelijk nood hebben aan actuele informatie. 

Applicaties blijken ook de \emph{voornaamste bron van informatie} te zijn voor gebruikers: Voor alle problemen, op spoorwijzigingen na, checken gebruikers hun smartphone. Voor spoorwijzigingen blijft informatie in de stations zelf, zoals omgeroepen informatie of digitale borden populairder. Ook wanneer informatie in de applicatie niet actueel is zoeken mensen hun toevlucht tot de omgeroepen informatie of digitale borden. Wanneer gebruikers gevraagd worden om informatiebronnen naar gebruik te rangschikken, blijven applicaties en websites ook hier bovenaan staan.

Wanneer gekeken wordt naar de \emph{tevredenheid}, blijkt dat applicaties hier uitzonderlijk goed scoren: meer dan 77\% van de gebruikers is hierover tevreden. Dit vormt een scherp contrast met websites, waar slechts 43\% tevreden over is. Voor alle informatiebronnen zijn er ontevreden gebruikers, wat er opnieuw op wijst dat er ruimte is voor verbetering.

Bij de respondenten zijn de \emph{besturingssystemen} Android en iOS gelijk vertegenwoordigd, met respectievelijk 39 en 40 respondenten. Een enkele respondent gebruikt Windows Mobile, nog een enkeling gebruikt Sailfish OS. 73\% gebruikt voornamelijk de officiële NMBS applicatie, de andere 27\% is ongeveer gelijk verdeeld over third-party applicaties, zoals HyperRail, iRail, Railer en BeTrains (tesamen goed voor 21\%) en een mix van algemene en buitenlandse applicaties, zoals citymapper, de Lijn en Deutsche Bahn. Het aandeel van third-party applicaties kan in dit onderzoek mogelijk beïnvloed zijn, gezien de enquête onder andere verspreid werd via distributiekanalen gelinkt aan iRail en deze third-party applicaties.

Deze applicaties worden \emph{vooral in stations} (95\% van de gebruikers), maar ook thuis (85\%) en op de trein zelf (80\%) gebruikt. Op de trein is men echter niet tevreden over de snelheid waarmee resultaten laden (66\% tevreden), thuis is men iets tevredener (72\% tevreden). Ook de gebruiksvriendelijkheid van opzoekingen daalt tijdens het reizen. 

Naar mogelijke oorzaken van deze dalingen tijdens een reis hoeft niet lang gezocht te worden: 60\% van de reizigers is ontevreden over het \emph{mobiele netwerk tijdens een treinreis}. Maar liefst 96\% van de reizigers heeft last van traag of niet ladende webpagina's, waarvan de meerderheid hier hier minstens de helft van de tijd hinder door ondervindt. Diezelfde mobiele dataverbinding is voor 53\% van de respondenten ook een bron van angst - ondanks dat er steeds meer data bij abonnementen geleverd wordt, zijn er nog steeds nog steeds mensen die schrik hebben om te veel data te verbruiken. Dit dient wel onmiddellijk genuanceerd te worden: 21\% maakt zich slechts een beetje zorgen. Deze groep zal vermoedelijk vooral voorzichtig zijn met media, en niet zozeer met het gebruik van routeplanning applicaties. Vooral jongeren (jonger dan 18) en personen ouder dan 35 jaar zijn voorzichtig met hun mobiele data.

Wanneer \emph{informatie niet opgezocht wordt via de app}, is dit voornamelijk omdat de gebruiker geen mobiele data heeft (33\%), omdat het opzoeken te lang duurt (24\%), of omdat informatie in het station handiger is. Gebruikers maken zich iets meer zorgen om het batterijgebruik van de applicatie (19\%) dan om het dataverbruik (15\%).

Wanneer gebruikers gevraagd worden naar \emph{wat ze belangrijk vinden in een applicatie} voor openbaar vervoer, komt het snel laden van resultaten overduidelijk op de eerste plaats. Dit wordt gevolgd door offline zoekopdrachten, waarna privacy, batterijgebruik en dataverbruik kort op elkaar volgen.

Ondanks dat \emph{privacy} een hot topic is, geeft 53\% van de gebruikers aan zich hier geen zorgen om te maken. Dit zouden ze misschien beter wel doen, want 75\% weet niet zeker of zijn of haar reisplannen over internet verstuurd worden, terwijl alle applicaties dit op dit moment doen. 12\% is er zelfs zeker van overtuigd dat zijn of haar reisplannen niet over internet verzonden worden. Ook over het versturen van onze locatie is men slecht geïnformeerd. Zo meent 17\% onterecht dat zijn of haar exacte locatie waarschijnlijk niet over internet verstuurd wordt. Voor third-party apps waarvan zeker geweten is dat ze de locatie niet over internet versturen, blijkt dan weer dat verschillende personen onterecht denken dat hun locatiegegevens toch verstuurd worden. 

Terwijl de meerderheid aangaf niet wakker te liggen van hun privacy bij routeplanning applicaties, blijkt toch dat het 85 en 77 percent van de respondenten zou storen moesten respectievelijk hun locatie en reisplannen over internet verstuurd worden.
 
Overstappen naar een andere applicatie is voor velen echter een brug te ver: slechts 35 en 37 percent van de respondenten zou overstappen naar een applicatie die respectievelijk hun reisplannen en locatie niet over internet verstuurd. Een ongeveer even groot aandeel geeft aan dat ze dit misschien zouden doen, afhankelijk van wat de alternatieven zijn. Deze aantallen dienen ook onmiddellijk genuanceerd te worden: ondanks recente privacyschandalen, blijft het overgrote deel van de smartphonegebruikers applicaties als Facebook Messenger en Whatsapp gebruiken. Overstappen en wennen aan een nieuwe applicatie kost moeite en tijd, wat gebruikers er mogelijk niet voor over hebben.

Tot slot werden gebruikers bevraagd naar \emph{wat ze vinden van een applicatie op basis van Linked Connections}. Twee personen gaven aan de uitleg niet volledig te begrijpen, en zijn van deze analyse uitgesloten.
Respondenten werden gevraagd om potentiële voordelen van Linked Connections van meest naar minst belangrijk te ordenen. Ondanks dat de gebruikers geïnformeerd werden dat Linked Connections volledige privacy biedt, en dat ongeveer een derde aan gaf indien mogelijk over te stappen naar een meer privacy-vriendelijke applicatie, komt privacy slechts bij 15\% van de bevraagden op de eerste plaats. Bij meer dan de helft eindigt privacy zelfs op de vierde plaats. Algemeen gezien blijft snelheid het belangrijkst, gevolgd door offline zoeken, aangepaste routes en tot slot privacy.

Dat aanpassen van routeplanning en privacy slechts op de vierde plaats komen wilt niet zeggen dat men dit niet belangrijk vindt. Tijdens user-tests gaven testers reeds aan deze rangschikkingen moeilijk te vinden, en wanneer gepolst wordt naar de interesse in afzonderlijke aspecten, blijkt dat mensen vooral het aanpassen van routeplanning en offline zoeken enorm interessant vinden, gevolgd door snelheid en privacy. Bij het aanpassen van routeplanning willen reizigers vooral kunnen zoeken naar routes met een kortere overstaptijd, drukke treinen mijden, en routes plannen met wat meer tijd om over te stappen. Privacy, wat op de laatste plaats staat, blijft interessant voor 83\% van de respondenten. Hieruit kan besloten worden dat Linked Connections enorm veel potentieel heeft voor mobiele applicaties. Deze resultaten zullen nog verder besproken worden in hoofdstuk~\ref{chap:interpretatie}.

\subsection{Conclusies op basis van de enquête}
Er wordt nu een antwoord geformuleerd op de in hoofdstuk~\ref{chap:onderzoek} gestelde vragen.
\begin{itemize}
	\item \textit{Biedt offline informatie een meerwaarde voor gebruikers?}\\
	Ja, gebruikers hebben grote interesse in offline opzoekingen, voornamelijk door een slechte mobiele internetverbinding tijdens het reizen, en in mindere mate omdat ze vrezen te veel data te verbruiken of gewoonweg niet over mobiel internet beschikken.
	\item \textit{Hecht de gebruiker belang aan privacy bij het gebruik van routeplanning apps? Zo ja, in welke mate?}\\
	De gebruiker hecht in beperkte mate belang aan zijn of haar privacy bij gebruik van routeplanning apps. De helft zegt zich hier geen zorgen over te maken, maar slechts een minderheid van de gebruikers weet welke data over internet verzonden worden. Een derde van de gebruikers zou overstappen naar apps die privacy-vriendelijker zijn.
	\item \textit{Heeft de gebruiker schrik om te veel mobiele data te verbruiken?}\\
	Ongeveer de helft van de gebruikers heeft schrik om te veel mobiele data te gebruiken, al maakt slechts een derde van de gebruikers zich hier ernstig zorgen om.
	\item \textit{Hecht de gebruiker belang aan dataverbruik bij het gebruik van routeplanning apps?}\\
	Één op zes gebruikers geeft aan soms geen informatie met een applicatie op te zoeken uit vrees te veel data te verbruiken. Wanneer gebruikers echter gevraagd wordt om een aantal aspecten van een routeplanning applicatie van belangrijk naar onbelangrijk te rangschikken, eindigt dataverbruik op de laatste plaats. Dataverbruik is voor de gebruiker dus van ondergeschikt belang aan de functionaliteit.
	\item \textit{Is de gebruiker tevreden met de snelheid van zijn huidige routeplanning app?}\\
	Thuis zijn zeven op tien gebruikers tevreden met de snelheid van routeplanning applicaties. Onderweg daalt dit tot een derde, hoogstwaarschijnlijk door slechte netwerkverbindingen.
	\item \textit{Wat is voor een gebruiker belangrijk in routeplanning apps?}\\
	Gebruikers vinden vooral het snel laden van resultaten belangrijk. Na snelheid volgen offline zoekopdrachten, waarna privacy, batterijgebruik en dataverbruik ongeveer even belangrijk zijn.
	\item \textit{Is de gebruiker geïnteresseerd in routeplanning op maat? Zo ja, welke aspecten spreken hem dan aan?}\\
	De gebruiker is zeer geïnteresseerd in routeplanning op maat, waarbij vooral het aanpassen van de overstaptijd in stations belangrijk is, en ook het mijden van drukke treinen als zeer interessant beschouwd wordt.
	\item \textit{Is de gebruiker geïnteresseerd in offline opzoekingen?}\\
	Zoals eerder vermeld vormen offline opzoekingen een meerwaarde voor gebruikers. Wanneer expliciet bevraagd, blijkt dan ook dat veel gebruikers hier in grote mate in geïnteresseerd zijn.
	\item \textit{Is de gebruiker geïnteresseerd in de mogelijke snelheid die Linked Connections biedt?}\\
	Ondanks dat veel gebruikers al tevreden zijn met de huidige snelheid van routeplanning applicaties, blijft het overgrote deel geïnteresseerd in het verder versnellen van deze applicaties.
	\item \textit{Is de gebruiker geïnteresseerd in de volledige privacy die Linked Connections biedt?}\\
	Ondanks dat privacy steevast onderaan de lijst met prioriteiten van gebruikers staat, wil dit niet zeggen dat gebruikers hierin niet geïnteresseerd zijn. Meer dan acht op tien gebruikers vindt dit interessant.
\end{itemize}

\section{Dataverbruik}
Zoals eerder aangehaald is op mobiele toestellen niet enkel de snelheid, maar ook andere zaken zoals batterijverbruik en connectiviteit van belang. Er wordt nu dieper ingegaan op het dataverbruik van de applicatie. 

\begin{figure}[ht]
	\centering
	\includegraphics[width=1.00\textwidth]{dataverbruik.eps}
	\caption[Dataverbruik per opzoeking]{Dataverbruik per type opzoeking en techniek}
	\label{fig:dataUsage}
\end{figure}

Wanneer dit dataverbruik gemeten wordt, zichtbaar in figuur~\ref{fig:dataUsage}, is te zien dat Linked Connections zoals te verwachten meer data gebruikt in de slechtste gevallen. Wat echter ook opvalt, is dat de mediaan steevast nul is. Dit is een rechtstreeks gevolg van de enorme cachebaarheid van Linked Connections. Linked Connections heeft ook last van grote uitlopers wanneer informatie wordt opgezocht met betrekking tot stations met weinig stops. In deze gevallen is het mogelijk dat Linked Connections alle vertrekken voor de komende paar dagen op moet halen, terwijl LC2Irail niet meer data dan anders moet versturen. LC2Irail zal immers steeds minstens één resultaat teruggeven, waardoor in het slechtste geval 10 verzoeken nodig zijn. Linked Connections kent geen harde bovengrens voor het aantal verzoeken, al is het zinloos om langer dan 15 seconden te zoeken~\citep{miller68}.

In het geval van Linked Connections zijn duidelijke verschillen te zien tussen het opzoeken van liveboards en andere types data. Dit is te verklaren door het feit dat om een liveboard van een station te reconstrueren vaak slechts één of twee pagina's nodig zijn. Hierdoor is minder data nodig, maar wordt ook minder data gecachet. Opzoekingen voor verschillende tijdstippen hebben verschillende data nodig, terwijl deze nog niet gecachet is, waardoor minder vaak de cache gebruikt kan worden. Bijgevolg hebben meer dan 25\% van de opzoekingen nieuwe data nodig.
Wanneer echter gekeken wordt naar dataverbruik voor routes en voertuigen, gebruiken deze verzoeken erg veel data. Hierdoor is er echt vaker een overlap tussen opzoekingen. Als gevolg wordt hier bij de enkele verzoeken enorm veel data binnengehaald, waarna vrijwel alle verzoeken uit cache beantwoord kunnen worden. Verder is te zien dat ook het derde kwartiel nul bedraagt, wat wilt zeggen dat meer dan 75\% van de verzoeken uit cache geladen kan worden. Dit laden is zichtbaar in de uitlopers, die tot 2,4 megabyte kunnen oplopen.

Door data vooraf te laden wanneer de gebruiker verbonden is via Wi-Fi, zouden ook deze uitlopers vermeden kunnen worden. In dit geval zou eerst een antwoord berekend kunnen worden op basis van offline (verouderde) gegevens, waarna alle gebruikte pagina's opnieuw opgehaald worden om het actuele antwoord op te bouwen.


\begin{figure}[ht]
	\centering
	\includegraphics[width=1.00\textwidth]{dataverbruik_werkelijk.eps}
	\caption[Werkelijk dataverbruik per opzoeking Linked Connections]{Dataverbruik in werkelijke omstandigheden per type opzoeking voor Linked Connections}
	\label{fig:dataUsageRealLife}
\end{figure}

Bij deze cijfers dient een belangrijke kanttekening gemaakt te worden. Deze gegevens tonen aan dat Linked Connections enorm cachebaar is, en enorm kan variëren in dataverbruik. Dit in tegenstelling tot een RPC API, die bijna altijd internet nodig heeft, met uitzondering van enkele populaire opzoekingen, die tijdens spitsuur uit cache geladen konden worden. Echter zijn deze gebaseerd op opzoekingen uitgevoerd op een server door verschillende gebruikers, en is de cache extreem afhankelijk van vorige opzoekingen. Het exacte dataverbruik wordt met andere woorden bepaald door het gebruikspatroon van de applicatie, en door eventueel gebruik van technieken zoals prefetching. Ter illustratie is in figuur~\ref{fig:dataUsageRealLife} het werkelijke dataverbruik tijdens online opzoekingen weergegeven (online opzoekingen zijn opzoekingen waarbij internet beschikbaar was, en de applicatie dus vrije keus had tussen cache en online data). In deze figuur is een veel realistischer beeld te zien, waarbij dat er, ondanks dat sommige opzoekingen volledig uit cache beantwoord kunnen worden, meestal toch data opgehaald wordt. Ook bij deze grafiek zijn er enkele kanttekeningen. Zo zijn de metingen voor liveboards en routes per incrementeel resultaat (er zijn dus mogelijk meerdere dergelijke opzoekingen nodig) en is het gebruik van de applicatie alsnog kunstmatig. Gebruikers kunnen in werkelijkheid mogelijk trager opzoeken, waardoor minder verzoeken uit cache beantwoord zouden kunnen worden.


\section{Batterijverbruik}
Naast dataverbruik is ook energieverbruik belangrijk bij draagbare toestellen. Er wordt nu dieper ingegaan op het batterijverbruik van de applicatie. Dit is moeilijk om exact te meten, gezien alle omstandigheden exact dezelfde moeten zijn. Automatische testen op UI vereisen een USB-verbinding, waarbij stroom via USB geleverd wordt in plaats van via de batterij. 

Om toch consistent verschillende applicaties te kunnen testen, wordt een variant op de testapplicatie gebruikt, waarin knoppen geplaatst worden om een korte benchmark uit te voeren. Hierdoor kunnen kan dezelfde test voor elk soort opzoeking uitgevoerd worden, waarna uit Android energiebeheer het verbruik afgelezen wordt en het toestel weer tot 100\% opladen alvorens de volgende test uit te voeren. Om aan testdata te komen wordt voor elk endpoint 5\% van de opzoekingen uit de iRail-logs gekozen. De resultaten van deze tests zijn te zien in figuur~\ref{fig:batteryUsage}.

\begin{figure}[ht]
	\centering
	\includegraphics[width=1.00\textwidth]{energieverbruik.eps}
	\caption[Energieverbruik per opzoeking]{Gemiddeld energieverbruik per type opzoeking en techniek. Telkens worden tien resultaten geladen, voldoende om het scherm van de gebruiker te vullen.}
	\label{fig:batteryUsage}
\end{figure}

Onmiddellijk vallen enkele verschillen tussen beide implementaties op. Zo is te zien dat Linked Connections steeds een veelvoud aan energie verbruikt. Dit varieert tussen dubbel zoveel voor Liveboards, tot acht maal zoveel voor routes. Hierbij is een duidelijk verband te zien tussen energieverbruik, de hoeveelheid gedownloade data en de complexiteit van de toegepaste algoritmes.
Liveboards, die weinig data en weinig processing vereisen, verbruiken het minst energie. Voertuigen, die meer data vereisen maar nog steeds een relatief eenvoudig algoritme hebben verbruiken ongeveer drie keer meer energie. Routes vereisen iets minder data dan voertuigen,  maar hebben een complexer algoritme. De processortijd blijkt een grote impact te hebben op de batterij, met een verdubbeling ten opzichte van het energieverbruik bij voertuigen tot gevolg.

Een belangrijke nuance bij deze resultaten is dat zelfs in het slechtste geval, het laden van routes via Linked Connections, slechts 4mAh verbruikt wordt. Dit is zeer weinig, en zal zelfs op smartphones met een zeer kleine batterij (2.000mAh) slechts 0,2\% batterij verbruiken. Op toestellen met een gemiddelde batterij (3.200mAh) zakt dit verder tot 0,125\%. Linked Connections verbruikt wel degelijk meer dan LC2Irail, maar het effect op de gebruiker blijft steeds beperkt.

\section{Beperkingen van dit onderzoek}
\label{sec:beperkingen}

Het grootste deel van deze masterproef is gebaseerd op user-testing en benchmarks van de ontwikkelde applicatie. Deze tests zijn echter aan enkele beperkingen onderhevig. Deze beperkingen hebben geen grote impact op het onderzoek en hebben geen invloed op de uiteindelijke conclusies, maar kunnen de volledigheid of precisie van het onderzoek lichtjes beïnvloeden.

\subsection{Kleine steekproef voor user-testing}
Zoals eerder vermeld ontbreekt op het moment van schrijven nog cruciale informatie in de Linked Connections server implementatie voor de NMBS, zoals of een voertuig al dan niet afgeschaft is, en op welk perron een voertuig aankomt. Hierdoor moesten teruggevallen worden op user-testing onder begeleiding, om gebruikers aan te sporen hun gebruikelijke opzoekingen te doen en te polsen naar hun ervaringen. Dit neemt relatief veel tijd in beslag, waardoor weinig mensen én zin, én tijd hebben. Voorts neemt deze methode van testen ook veel tijd in beslag voor de onderzoeker. 

De groep testgebruikers is wel gevarieerd, zowel in persoonlijke eigenschappen zoals leeftijd, als in reisgewoontes per trein. Wanneer de gehele testgroep duidelijk de voorkeur geeft aan een bepaalde variant, kan deze keuze veralgemeend worden naar de gehele populatie. Wanneer er echter geen grote meerderheid voor eenzelfde variant kiest, moeten echter opgelet worden met het trekken van conclusies.

\subsection{Beperkt aantal unieke toestellen getest}
Uit de voorgaande secties blijkt dat het gebruikte toestel van groot belang is voor de prestaties van de lokale Linked Connections implementatie. Tijdens het user-testen werd gebruik gemaakt van twaalf verschillende smartphones. Dit aantal is relatief beperkt in vergelijking met het aanbod op de huidige smartphonemarkt. Eventuele verder onderzoek zal de prestaties van Linked Connections op verschillende toestellen moeten vastleggen.

\subsection{Processorgebruik is niet exact meetbaar}
De Android CPU-Profiler beïnvloedt de prestaties van de applicatie zodanig dat het onmogelijk is om een correct beeld te krijgen van het processorgebruik. Er kan een beeld gevormd worden welke onderdelen van de applicatie het meest processortijd vragen, maar exacte tijdsmetingen zijn niet mogelijk. Deze problemen worden ook door andere Android ontwikkelaars op internet beschreven\footnote{\url{https://stackoverflow.com/questions/49555983/background-concurrent-copying-gc-freed}}. Deze problemen treden op door de nieuwe Android CPU profiler, die zelf teveel processortijd op het apparaat vereist.

\subsection{Prestaties zijn sterk afhankelijk van implementatiedetails}
Zoals blijkt uit grafieken \ref{fig:liveboardlabtest}, \ref{fig:routelabtest} en \ref{fig:vehiclelabtest} is de performantie van de lokale Linked Connections implementatie sterk afhankelijk van details in de implementatie - Het is dus niet enkel belangrijk om de algoritmes te optimaliseren, maar ook om rekening te houden met processen zoals Garbage Collection. Dit werd pas in een gevorderd stadium van de proef vastgesteld. Het is mogelijk dat de resultaten in dit onderzoek nog verder verbeterd kunnen worden door dezelfde algoritmes efficiënter te implementeren.

\subsection{Kleine afwijkingen tussen opzoekingen LC en LC2Irail}
Hoewel zowel Linked Connections en LC2Irail betrouwbaar werken in de testapplicatie, zijn er nog een beperkt aantal gevallen waarin het automatisch laden van incrementele resultaten niet werkt. Dit is onder andere mogelijk bij stations waar voor langer perioden geen voertuig stopt, of routes waarvoor slechts enkele resultaten per dag beschikbaar zijn. Om te zorgen dat ook voor deze opzoekingen incrementele resultaten feilloos laden zijn verdere verfijningen nodig aan de stopvoorwaarden en implementatie. Zo moet voorkomen worden dat er bij opzoekingen die data van meerdere dagen vereisen te veel callbacks gebruikt worden, een neveneffect dat tijdens de ontwikkeling nog niet bekend was. Aangezien enkel succesvolle opzoekingen meegenomen worden in testresultaten, is het hierdoor mogelijk dat er een klein verschil zit op het aantal meetpunten bij vergelijkende tests tussen LC en LC2Irail. Dit verschil treed vooral op bij het opzoeken van routes, en heeft geen merkbare invloed op resultaten door de grootte van de steekproef.