\section{Overzicht}
\label{sec:impl_overzicht}
Om kort een overzicht te schetsen, zal deze sectie de implementatie overlopen en hierbij beschrijven wat gemaakt is en wat nog te doen is. Bij de vorige secties werden de functionaliteiten besproken in volgorde van hoe ze geïmplementeerd werden. In dit overzicht zullen ze besproken worden in volgorde van hoe ze uitgevoerd worden. Er kan allesinds wel besloten worden dat een complete implementatie van GeoSPARQL zeer veel werk vraagt. Hier is al een deel van gemaakt, maar hier moet nog veel aan gewerkt worden.

\subsection{Huidige status}
Bij deze implementatie wordt eerst gecontroleerd of het gaat over een GeoSPARQL functionaliteit. Wanneer dit het geval is, zal eerst de linker- en rechterparameter recursief opgelost worden, zodat de functionaliteit effectief opgelost kan worden. Bij het oplossen van deze functionaliteit zal eerst de ``WKT string'' omgezet worden naar een GeoJSON object, waarbij gecontroleerd wordt of er een projectie (= referentiestelsel) meegegeven is (indien niet wordt gewerkt met de standaard waarde, genaamd ``WGS84''). Indien beide parameters van de functie niet dezelfde projectie zouden hebben, dan worden de coördinaten van de tweede parameter omgerekend naar de respectievelijke waarden van de projectie van de eerste parameter. Eenmaal dat dit gedaan is kan de functie eenvoudig uitgewerkt worden met behulp van ``Turf.js''. Dit resultaat wordt dan teruggegeven, zodat de boomstructuur dit kan gebruiken voor de verdere uitwerking van de recursieve structuur. Voor niet-topologische functies kan het nodig zijn om een GeoJSON object terug te geven. OM een uniforme werking van sparqlee te kunnen behouden, wordt dit GeoJSON object opnieuw geserialiseerd naar WKT formaat. 

\subsection{Toekomstwerk}
\label{subsec:toekomstwerk}
Aangezien deze implementatie slechts een beperkte implementatie van GeoSPARQL is, zal hier in de toekomst nog verder aan gewerkt moeten worden. Er wordt momenteel slechts in zeer beperkte mate ondersteuning aangeboden voor het gebruik van ``Feature''. Er is ook enkel ondersteuning voor ``WKT strings'' en niet voor het alternatieve ``GML'' formaat. Vervolgens moet er een aanvulling gemaakt worden van zowel de topologische functies als de niet-topologische functies. Bij de topologische functies is enkel gekeken naar de ``Simple Features'' familie (en deze is ook niet helemaal compleet), maar er zijn nog twee andere families. Bij de niet-topologische functies zijn er slechts twee functies geïmplementeerd, zijnde de ``union'' en de ``intersection'' functies. Hier zijn er nog verscheidene andere functies. Ten slotte moet het mogelijk zijn om deze functies te gebruiken in de vorm van predikaat. Hiervoor moet er een functionaliteit zijn voor het herschrijven van de queries. Deze is totaal niet gebruikt, maar is wel nodig voor een volledige implementatie van GeoSPARQL. 