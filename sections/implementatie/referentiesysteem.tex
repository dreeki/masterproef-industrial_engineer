\section{Referentiesysteem}
\label{sec:projecties}
Voor het bekomen van een werkende implementatie is er nog een volgende functionaliteit die ondersteund moet worden. Zo moet het nog mogelijk zijn om te werken met verschillende Referentiesystemen. Om nog eens te benadrukken waarom dit belangrijk is, wordt dit geïllustreerd met een voorbeeld. Verschillende landen of zelfs delen van landen liggen op andere aardplaten die los van elkaar bewegen of zelfs tegen elkaar botsen. Hierdoor kunnen landen ten opzichte van elkaar bewegen. Het best voorbeeld hiervan is Australië dat jaarlijks wat kan verplaatsen. Op deze manier zou na enkele jaren elke geografische entiteit opnieuw vastgelegd moeten worden. Hiervoor wordt gebruik gemaakt van aparte referentiesystemen. Dit is een manier om een punt te beschrijven in coördinaten ten opzichte van een relatief punt. Om echter coördinaten in verschillende referentiesystemen te kunnen vergelijken met elkaar, moeten deze eerst omgerekend worden naar hetzelfde referentiesysteem.

Voor het oplossen van dit probleem zijn er twee mogelijkheden. Ofwel worden beide referentiesystemen omgerekend naar één op voorhand gedefinieerd referentiesysteem, ofwel wordt één van beide referentiesystemen naar de andere omgerekend. Er is gekozen voor de tweede optie voor twee redenen. In dit geval moet voor elk koppel vormen dat vergeleken wordt slechts één vorm herrekend worden, wat zorgt voor een betere performantie. De tweede reden is dat het OGC zelf voorschrijft dat gewerkt moet worden in het referentiesysteem van de eerste vorm.

\subsection{Proj4js}
Proj4js is een \textit{library} die zorgt voor het transformeren van coördinaten van het ene referentiesysteem naar coördinaten van het andere referentiesysteem. Hierbij zijn al enkele projecties op voorhand gedefinieerd binnen Proj4js, maar het is ook mogelijk om zelf nieuwe projecties toe te voegen. Proj4js is dan ook de gebruikte \textit{library} voor het uitvoeren van deze berekening bij de gemaakte implementatie.

\subsection{Beperkingen}
Bij deze functionaliteit zijn er echter wel nog beperkingen. Zo zijn er nog niet veel referentiesystemen beschreven binnen GeoSPARQL, zodat er slechts in beperkte mate de mogelijkheid is om hiervan gebruik te maken. De architectuur van hoe hiermee gewerkt wordt is echter wel in orde. Kortom betekent dit dat het nog niet zeer handig is om te gebruiken, maar dat het wel klaar is om in toekomstig werk toegepast te worden.